%
% My own commands, commands adapted from Joey Paquet and Peter Grogono
% Serguei Mokhov, mokhov@cs.concordia.ca
%
% $Id: styles.tex,v 1.4 2005/07/03 22:59:15 mokhov Exp $
%

\newcommand{\marf}{\textsf{MARF}}


% Cross-reference commands.
% Per Dr. Grogono and my own self.
\newcommand{\xf}[1]{Figure~\ref{#1}}
\newcommand{\xp}[1]{page~\pageref{#1}}
\newcommand{\xs}[1]{Section~\ref{#1}}
\newcommand{\xa}[1]{Appendix~\ref{#1}}
\newcommand{\xc}[1]{Chapter~\ref{#1}}
\newcommand{\xt}[1]{Table~\ref{#1}}

%\newcommand{\bestscore}[1]{$#1\%$}
\newcommand{\bestscore}[1]{\textsf{{\large #1\%}}}

%
% Abbrs
%

\newcommand{\rpc}{{RPC\index{RPC}}}
\newcommand{\rmi}{{RMI\index{RMI}}}
\newcommand{\gnu}{{GNU\index{GNU}}}
\newcommand{\tcpip}{{TCP/IP\index{TCP/IP}}}
\newcommand{\gipsy}{{GIPSY\index{GIPSY}}}

%
% The Imperatives
%

\newcommand{\C}{{C\index{C}}}
\newcommand{\cpp}{{C++\index{C++}}}
\newcommand{\perl}{{Perl\index{Perl}}}
\newcommand{\java}{{Java\index{Java}}}
\newcommand{\python}{{Python\index{Python}}}

%
% The Functionals
%

\newcommand{\lisp}{{LISP\index{LISP}}}
\newcommand{\scheme}{{Scheme\index{Scheme}}}
\newcommand{\haskell}{{Haskell\index{Haskell}}}

%
% Util
%

\newcommand{\tab}[1]{\hspace{#1pt}}

\newcommand{\shrule}[0]{\vspace{3pt}\hrule\vspace{6pt}}
\newcommand{\ehrule}[0]{\vspace{6pt}\hrule\vspace{3pt}}

\newcommand{\nonterminal}[1]{$\mathtt{<\!\!#1\!\!>}$}

\newcommand{\source}[1]
{
	{\shrule}
	\scriptsize
	#1
	\normalsize
	\hrule
}

\newcommand{\sourcefloat}[3]
{
	\begin{figure}[t]
	\begin{centering}
	\begin{minipage}[b]{0.5\textwidth}
	\source{#1}
	\end{minipage}
	\caption{\small{#3}}
	\label{#2}
	\end{centering}
	\end{figure}
}

\newcommand{\todo}[0]
{
	{\Large \[TODO\]}
}

\newcommand{\file}[1]{\texttt{#1}\index{Files!#1}}
\newcommand{\tool}[1]{\texttt{#1}\index{Tools!#1}}
\newcommand{\api}[1]{\texttt{#1}\index{API!#1}}

%
% Def
%

\newcommand{\statement}[2]
{
	\vspace{7pt}
	\shrule
	{\bf #1}

	#2
	\ehrule
	\vspace{7pt}
}

\newcommand{\proposition}[2]
{
	\statement{Proposition #1}{#2}
}

\newcommand{\definition}[2]
{
	\statement{Definition #1}{#2}
}

\newcommand{\axiom}[2]
{
	\statement{Axiom #1}{#2}
}

\newcommand{\theorem}[2]
{
	\statement{Theorem #1}{#2}
}
