\chapter{Sample Data and Experimentation}
\index{Sample Data}
\index{Experiments}

$Revision: 1.21 $

\section{Sample Data}
\index{Sample Data}

$Revision: 1.16 $

We have both female and male speakers, with age
ranging from a college student to a University professor.
The table \ref{tab:samples} has a list of people who have contributed their
voice samples for our project (with the first four being ourselves).
We want to thank them once again for helping us out.

\begin{table}
\begin{minipage}[b]{\textwidth}
\centering
\begin{tabular}{|c|c|c|c|} \hline
ID          & Name               & Training Samples & Testing Samples\\ \hline\hline
1           & Serge              & 14               & 1 \\
2           & Ian                & 14               & 1 \\
3           & Steve              & 12               & 3 \\
4           & Jimmy              & 14               & 1 \\
5           & Dr. C.Y. Suen      & 2                & 1 \\
6           & Margarita Mokhova  & 14               & 1 \\
7           & Alexei Mokhov      & 14               & 1 \\
8           & Alexandr Mokhov    & 14               & 1 \\
9           & Graham Sinclair    & 12               & 2 \\
10          & Jihed Halimi       & 2                & 1 \\
11          & Madhumita Banerjee & 3                & 1 \\
13          & Irina Dymova       & 3                & 1 \\
14          & Aihua Wu           & 14               & 1 \\
15          & Nick               & 9                & 1 \\
16          & Michelle Khalife   & 14               & 1 \\
17          & Shabana            & 7                & 1 \\
18          & Van Halen          & 8                & 1 \\
19          & RHCP               & 8                & 1 \\
20          & Talal Al-Khoury    & 14               & 1 \\
21          & Ke Gong            & 14               & 1 \\
22          & Emily Wu Rong      & 14               & 1 \\
23          & Emily Ying Lu      & 14               & 1 \\
24          & Shaozhen Fang      & 14               & 1 \\
25          & Chunlei He         & 14               & 1 \\
26          & Shuxin Fan         & 15               & 1 \\
27          & Shivani Bhat       & 14               & 1 \\ \hline\hline
{\bf Total} & {\bf 25}           & {\bf 293}        & {\bf 29} \\ \hline
\end{tabular}
\end{minipage}
\caption{Speakers contributed their voice samples.}
\label{tab:samples}
\end{table}

% EOF


\section{Comparison Setup}

The main idea was to compare combinations (in {\marf}: {\it configurations})
of different methods and variations within them in terms of recognition
rate performance. That means that having several preprocessing modules, several feature
extraction modules, and several classification modules, we can (and did)
try all their possible combinations.

That includes:

\begin{enumerate}
	\item Preprocessing: No-filtering, normalization, low-pass, high-pass,
	      band-pass, and high-frequency boost, high-pass and boost filters,
	      and endpointing.
	\item Feature Extraction: FFT/LPC/Min-Max/Random algorithms comparison.
	\item Classification: Distance classifiers, such as Chebyshev, Euclidean,
	      Minkowski, Mahalanobis, and Diff distances, as well as Neural Network and Random
	      classification.
\end{enumerate}


For this purpose we have written a \api{SpeakerIdentApp}, a command-line application
(so far, but GUI is planned) for TI speaker identification. We ran it for every possible configuration
with the following shell script, namely \tool{testing.sh}:

\vspace{15pt}
\hrule
{\scriptsize \input{testing-sh}}
\hrule
\vspace{15pt}

The above script is for Linux/UNIX environments. To run a similar script
from Windows, use \tool{testing.bat} for classification and the \tool{retrain}
shortcut for re-training and classification. These have been completed
during the development of the 0.3.0 series.

See the results section (\ref{sect:results}) for results analysis.

\section{What Else Could/Should/Will Be Done}

There is a lot more that we realistically could do, but due to lack of time, these things
are not in yet. If you would like to contribute, let us know, meanwhile we'll keep working
at our speed.

\subsection{Combination of Feature Extraction Methods}

For example, assuming we use a combination of LPC coefficients and F0
estimation, we could compare the results of different combinations of
these, and discuss them later. Same with the
Neural Nets (modifying number of layers and number or neurons, etc.).

We could also do a 1024 FFT analysis and compare it against a 128 FFT
analysis.  (That is, the size of the resulting feature vector would be 512 or 64 respectively).
With LPC, one can specify the number of coefficients you want, the more you
have the more precise the analysis will be.

\subsection{Entire Recognition Path}

The \api{LPC} module is used to generate a mean vector of LPC coefficients for
the utterance. \api{F0} is used to find the average fundamental frequency of the
utterance. The results are concatenated to form the output vector, in a
particular order. The classifier would take into account the weighting of
the features: Neural Network would do so implicitly if it benefits the speaker
matching, and stochastic can be modified to give more weight to the F0 or
vice versa, depending on what we see best (i.e.: the covariance matrix in the
Mahalanobis distance (\ref{sect:mahalanobis})).

\subsection{More Methods}

Things like $F_0$, stochastic, and some other methods have not made to this release.
More detailed on this aspect, please refer to the TODO list in the Appendix.

% EOF
