\subsection{Feature Extraction Aggregation}
\label{sect:aggregator}
\index{Feature Extraction Aggragation}
\index{Feature Extraction!Feature Extraction Aggragation}

$Revision: 1.2 $

\subsubsection{Description}

This method appeared in {\marf} as of 0.3.0.5.
This class by itself does not do any feature extraction, but
instead allows concatenation of the results of several actual feature
extractors to be combined in a single result. This should give the
classification modules more discriminatory power (e.g. when combining
the results of FFT and F0 together).
\api{FeatureExtractionAggregator} itself still implements
the \api{FeatureExtraction} API in order to be used in the main
pipeline of \api{MARF}.

The aggregator expects \api{ModuleParams} to be set to the
enumeration constants of a module to be invoked followed by that module's
enclosed instance \api{ModuleParams}. As of this implementation,
that enclosed instance of \api{ModuleParams} isn't really used, so
the {\bf main limitation} of the aggregator is that all the aggregated
feature extractors act with their default settings. This will happen until
the pipeline is re-designed a bit to include this capability.

The aggregator clones the incoming preprocessed sample for each
feature extractor and runs each module in a separate thread. A the
end, the results of each tread are collected in the same order as
specified by the initial \api{ModuleParams} and returned
as a concatenated feature vector. Some meta-information is available
if needed.

\subsubsection{Implementation}

Class: \api{marf.FeatureExtraction.FeatureExtractionAggregator}.

% EOF
