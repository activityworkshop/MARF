\subsection{Diff Distance}
\label{sect:diff-distance}
\index{Distance!Diff}
\index{Classification!Diff Distance}

$Revision: 1.2 $

\subsubsection{Summary}

\begin{itemize}
\item Implementation: \api{marf.Classification.Distance.DiffDistance}
\item Depends on: \api{marf.Classification.Distance.Distance}
\item Used by: \api{test}, \api{marf.MARF}, \api{SpeakerIdentApp}
\end{itemize}

\subsubsection{Theory}

When Serguei Mokhov invented this classifier in May 2005, the original idea
was based on the way the \tool{diff} UNIX utility works. Later, for
performance enhancements it was modified. The essence of the diff distance is to
count how one input vector is different from the other in terms of
elements correspondence. If the Chebyshev distance between the two
corresponding elements is greater than some error $e$, then this
distance is accounted for plus some additional distance penalty $p$
is added. Both factors $e$ and $p$ can vary depending on desired
configuration. If the two elements are equal or pretty close
(the difference is less than $e$) then a small ``bonus'' of $e$ is subtracted
from the distance.

$$ d(x,y) = \sum_{i}{|x_i-y_i| + p, \mathit{if} |x_i-y_i| > e, \mathit{or} (-e)} $$

\noindent
where $x$ and $y$ are feature vectors of the same length.

% EOF
