\subsection{TestWaveLoader}
\index{Applications!TestWaveLoader}
\index{Testing Applications!TestWaveLoader}

$Revision: 1.5 $

\api{TestWaveLoader} is one of the testing applications in {\marf}. It tests functionality of
the \api{WAVLoader} of {\marf}.

From user's point of view, \api{TestWaveLoader} provides with usage, input wave sample, output wave
sample, and output textual file command-line options.
By entering \texttt{--help}, or \texttt{-h}, or
when there are no arguments, the usage information will be displayed.
It explains the arguments used in \api{TestWaveLoader}. The first argument is an input wave sample file whose name is a mandatory argument.
The second and the third arguments are the output wave sample file and output textual sample file of the loaded input sample. The names of the output
files are optional arguments. If user does not provide any or both of the last two arguments, the output files will be saved in the files provided by 
\api{TestWaveLoader}.

\noindent
Complete usage information:

\vspace{15pt}
\hrule
\begin{verbatim}

Usage:
  java TestWaveLoader --help | -h

     displays usage information and exits

  java TestWaveLoader --version

     displays application and MARF versions

  java TestWaveLoader <input-wave-sample> [ <output-wave-sample> [ <output-txt-sample> ] ]

     loads a wave sample from the <input-wave-sample> file and
     stores the output wave sample in <output-wave-sample> and
     its textual equivalent is stored in <output-txt-sample>. If
     the second argument if omitted, the output file name is
     assumed to be "out.<input-wave-sample>". Likewise, if
     the third argument if omitted, the output file name is
     assumed to be "<output-wave-sample>.txt".

\end{verbatim}

\hrule
\vspace{15pt}

The application is made to exercise the following {\marf} modules.
The main module is the \api{WAVLoader} in the \api{marf.Storage.Loaders} package.
the \api{Sample} and the \api{SampleLoader} modules in the \api{marf.Storage} help \api{WAVLoader} prepare loading wave files.
\api{Sample} maintains and processes incoming sample data. \api{SampleLoader} provides sample loading interface for
all the {\marf} loaders. It must be overridden by a concrete sample loader. 
For loading wave samples, \api{SampleLoader} needs \api{WAVLoader} implementation.
Three modules work together to load in and write back a wave sample whose name was
provided by users in the first argument, to save the loaded sample into a newly-named wave file,
to save loaded input data into a data file referenced by \texttt{oDatFile}, 
and to output sample's duration and size to STDOUT.

As we know, the output of \api{TestWaveLoader} saves the loaded wave sample in a newly named output wave file.
Its output also saves the input file data into a textual file.
\api{TestWaveLoader} gives both users and programmers direct information of the results of MARF wave loader.
The output sample can be compared with the expected output in the \file{expected} folder to detect any errors.

% EOF
