\subsection{TestLPC}
\index{Applications!TestLPC}
\index{Testing Applications!TestLPC}

$Revision: 1.5 $

The \api{TestLPC} application targets
the LPC unit testing as a preprocessing module.
Through a number of options it also allows choosing
between two implemented loaders -- \api{WAVLoader}
and \api{SineLoader}. To faciliate option processing
\api{marf.util.OptionProcessor} used that provides
an ability of maintaining and validating valid/active
option sets. The application also utilizes the \api{Dummy}
preprocessing module to perform the normalization of
incoming sample.

The application supports the following set of options:

\begin{itemize}
\item
\texttt{--help} or \texttt{-h} cause the application to
display the usage information and exit. The usage information
is also displayed if no option was supplied.

\item
\texttt{--sine} forces the use of \api{SineLoader} for
sample data generation. The output of this option is also
saved under \file{expected/sine.out} for regression testing.

\item
\texttt{--wave} forces the application to use the \api{WAVLoader}.
This option requires a mandatory filename argument of a wave file
to run the LPC algorithm against.
\end{itemize}

\noindent
Complete usage information:

\vspace{15pt}
\hrule
\begin{verbatim}

Usage:
    java TestLPC --help | -h
        displays this help and exits

    java TestLPC --version
        displays application and MARF versions

    java TestLPC --sine
        runs the LPC algorithm on a generated sine wave

    java TestLPC --wave <wave-file>
        runs the LPC algorithm against specified voice sample

\end{verbatim}

\hrule
\vspace{15pt}

% EOF
