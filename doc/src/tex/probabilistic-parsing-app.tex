%
% $Id: probabilistic-parsing-app.tex,v 1.29 2006/02/26 23:38:46 mokhov Exp $
%

\subsection{CYK Probabilistic Parsing with \api{ProbabilisticParsingApp}}
\index{CYK Probabilistic Parsing}
\index{CYK Probabilistic Parsing!Application}
\index{Applications!CYK Probabilistic Parsing}

$Revision: 1.29 $

Originally written on April 12, 2003.

\subsubsection{Introduction}

This section describes the implementation of the CYK probabilistic parsing
algorithm \cite{cyk} implemented in {\java} and discusses the experiments, grammars used,
the results, and difficulties encountered.

\subsubsection{The Program}

\paragraph{Manual and the Source Code}

Mini User Manual along with instructions on how to run the application are provided
in the \xs{sect:prob-manual}. The source code is provided in the electronic form only with few extracts
in the presented in the document.

\paragraph{Grammar File Format}

Serguei has developed a ``grammar compiler'' for the Compiler Design course, and
we have adapted it to accept probabilistic grammars. The grammar compiler reads a source
grammar text file and compiles it (some rudimentary lexical and semantic checks are in place).
As a result, a successfully ``compiled'' \api{Grammar} object
has a set of terminals, non-terminals, and rules stored in a binary file.
Parsers re-load this compiled grammar and do the main parsing of what they are supposed to parse.

\begin{figure}
\hrule\vskip4pt
\begin{verbatim}
<LHS> ::= PROBABILITY RHS %EOL

#  single-line comment; shell style

// single-line comment; C++ style

/*
 * multi-line comment; C style
 */
\end{verbatim}
\caption{Grammar File Format}
\label{fig:grammar-format}
\vskip4pt\hrule
\end{figure}


The grammar file for the grammar file has the format presented in \xf{fig:grammar-format}. Description
of the elements is below. The example of the grammar rules is in \xf{fig:grammar-example}.
{\bf Whenever one changes the grammar, it has to be recompiled to take effect.}

\begin{itemize}

\item
    \verb+<LHS>+ is a single non-terminal on the left-hand side of the rule.

\item
    \verb+::=+ is a rule
    operator separating LHS and RHS.

\item
    \verb+PROBABILITY+ is a floating-point number indicating rule's
    probability.

\item
    \verb+RHS+ for this particular assignment has to be in CNF, i.e. either \verb+<B> <C>+
    or \verb+terminal+ with \verb+<A>+ and \verb+<B>+ being non-terminals.

\item
    All non-terminals have to be enclosed within the angle brackets \verb+<+ and \verb+>+.

\item
    All grammar rules have to terminated by \verb+%EOL+ that acts a semicolon in {\C}/{\cpp} or
    {\java}. It indicates where to stop processing current rule and look for the next
    (in case a rule spans several text lines).

\item
    Amount of white space between grammar elements doesn't matter much.

\item
    The grammar file has also a notion of comments. The grammar compiler accepts
    shell-like single line comments when lines start with \verb+#+ as well as {\C} or {\cpp} comments
    like \verb+//+ and \verb+/* */+ with the same effect as that of {\C} and {\cpp}.

\end{itemize}


\begin{figure}
\hrule\vskip4pt
\begin{verbatim}
/*
 * 80% of sentences are noun phrases
 * followed by verb phrases.
 */
<S> ::= 0.8 <NP> <VP> %EOL

// A very rare verb
<V> ::= 0.0001 disambiguate %EOL

# EOF
\end{verbatim}
\caption{Grammar File Example}
\label{fig:grammar-example}
\vskip4pt\hrule
\end{figure}


\paragraph{Data Structures}

The main storage data structure is an instance of the \api{Grammar} class that holds
vectors of
terminals (see file \file{marf/nlp/Parsing/GrammarCompiler/Terminal.java}),
non-terminals (see file \file{marf/nlp/Parsing/GrammarCompiler/NonTerminal.java}),
and rules (see file \\\file{marf/nlp/Parsing/GrammarCompiler/ProbabilisticRule.java}).

While the grammar is being parsed and compiled there are also
various grammar tokens and their types involved.
Since they are
not very much relevant to the subject of this application we won't talk about them
(examine the contents of the \file{marf/nlp/Parsing/} and \file{marf/nlp/Parsing/GrammarCompiler/}
directories if you care).

The CYK algorithm's data structures, the $\pi$ and \verb+back+ arrays,
are represented as 3-dimensional array of doubles and vectors of back-indices
respectively: \verb+double[][][] oParseMatrix+ and \verb+Vector[][][] aoBack+
in \file{marf/nlp/Parsing/ProbabilisticParser.java}. There is also a vector of
words of the incoming sentence to be parsed, \verb+Vector oWords+.


%%%%%%%%%%%%%%%%%%%%%%%%%%%%%%%%%%%%%%%%%%%%%%%
%%%%%%%%%%%%%%%%%%%%%%%%%%%%%%%%%%%%%%%%%%%%%%%

\subsubsection{Methodology}

We have experimented (not to full extent yet) with three grammars:
{\em Basic}, {\em Extended}, and {\em Realistic}. Description of the grammars and how
they were obtained is presented below. The set of testing sentences was initially
based on the given (Basic) grammar to see whether the CYK algorithm indeed
parses all syntactically valid sentences and rejects the rest. Then the
sentence set was augmented from various sources (e.g. the paper Serguei presented
and on top of his head). Finally, the original requirements were attempted to be used
as a source of grammar.

\paragraph{Basic Grammar}

The basic grammar given in the requirements has been
used at first to develop and debug the application.
The basic grammar is in \xf{fig:basic-grammar}.

\begin{figure}
\hrule\vskip4pt
\tiny
\input{grammar.original.tex}
\normalsize
\caption{Original Basic Grammar}
\label{fig:basic-grammar}
\vskip4pt\hrule
\end{figure}

\paragraph{Extended Grammar}

The basic grammar was extended with few rules
from \cite{jurafsky}, p. 449.
The probability scores are {\bf artificial} and
adapted from the basic grammar and the book's grammar, and
recomputed {\em approximately}\footnote{Preserving proportional relevance from all sources
and making sure they all add up to 100\%} by hand.
The extended grammar is in \xf{fig:extended-grammar}.
Both basic and extended grammars used the same set of
testing sentences presented in \xf{fig:test-sentences}.
There is a number of sentences for which we have never come up
with rules as initially intended, so there are no parses
for the them in the output can be seen yet, a TODO.

\begin{figure}
\hrule\vskip4pt
\tiny
\input{grammar.extended.tex}
\normalsize
\caption{Extended Grammar}
\label{fig:extended-grammar}
\vskip4pt\hrule
\end{figure}

\begin{figure}
\hrule\vskip4pt
\scriptsize
my rabbit has a white smile

\par
my rabbit has a smile

\par
my rabbit has a telephone

\par
rabbit my a white has smile

\par
a slim blue refrigerator jumped gracefully out of the bottle

\par
my smile has a rabbit

\par
the cat eats my white rabbit

\par
a white smile eats the cat

\par
my cat has a white rabbit

\par
my cat has white rabbit

\par
cat has a white rabbit

\par
smile has my cat

\par
can you book TWA flights

\par
the lion jumped through the hoop

\par
the trainer jumped the lion through the hoop

\par
the butter melted in the pan

\par
the cook melted the butter in the pan

\par
the rich love their money

\par
the rich love sometimes too

\par
the contractor built the houses last summer

\par
the contractor built last summer

\par
\normalsize

\caption{Input sentences for the Basic and Extended Grammars}
\label{fig:test-sentences}
\vskip4pt\hrule
\end{figure}

\paragraph{Realistic Grammar}\label{sect:realistic-grammar}

Without having 100\% completed the extended grammar, we jumped
to develop something more ``realistic'', the Realistic Grammar.
Since the two previous grammars are quite artificial, to test
out it on some more ``realistic'' data we came up with the
best grammar we could
from the sentences of the requirements (other than
the sample grammar; the actual requirements were used).
The sentences, some were expanded, are in \xf{fig:asmt-sentences},
and the grammar itself is in \xf{fig:asmt-grammar}. Many of the rules
of the form of \verb+A->BC+ still may not have truly correct probabilities
corresponding to the paper due to the above two reasons.

\begin{figure}
\hrule\vskip4pt
\scriptsize
you should submit a paper listing and report and an electronic version of your code

\par
implement the CYK algorithm to find the best parse for a given sentence

\par
your program should take as input a probabilistic grammar and a sentence and display the best parse tree along with its probability

\par
you are not required to use a specific programming language

\par
Perl, CPP, C or Java are appropriate for this task

\par
as long as you explain it, your grammar can be in any format you wish

\par
your grammar can be in any format you wish as long as you explain it

\par
experiment with your grammar

\par
is it restrictive enough

\par
does it refuse ungrammatical sentences

\par
does it cover all grammatical sentences

\par
write a report to describe your code and your experimentation

\par
your report must describe the program

\par
your report must describe the experiments

\par
describe your code itself

\par
indicate the instructions necessary to run your code

\par
describe your choice of test sentences

\par
describe your grammar and how you developed it

\par
what problems do you see with your current implementation

\par
what problems do you see with your current grammar

\par
how would you improve it

\par
both the code and the report must be submitted electronically and in paper

\par
in class, you must submit a listing of your program and results and the report

\par
through the electronic submission form you must submit the code of your program and results and an electronic version of your report

\par
\normalsize

\caption{Requirements' sentences used as a parsed ``corpus''}
\label{fig:asmt-sentences}
\vskip4pt\hrule
\end{figure}

\begin{figure}
\vskip4pt\hrule\vskip4pt
\tiny
\input{grammar.asmt.tex}
\normalsize
\caption{Realistic Grammar}
\label{fig:asmt-grammar}
\vskip4pt\hrule
\end{figure}

\paragraph{Grammar Restrictions}

All incoming sentences, though case-sensitive, are required to be
in the lower-case unless a given word is a proper noun or the
pronoun {\it I}. The current system doesn't deal with punctuation
either, so complex sentences that use commas, semicolons, and
other punctuation characters may not (and appear not to) be
parsed correctly (such punctuation is simply ignored).

However, all the above restrictions can be solved just at the
grammar level without touching a single line of code,
but they were not dealt with yet in the first prototype.

\paragraph{Difficulties Encountered}
\label{sect:prob-app-troubles}

\paragraph*{Learning Probabilistic Grammars}

The major problem of this type of grammars is to learn them.
This requires at least having a some decent POS tagger
(e.g. in \cite{brill}) and decent knowledge of the English grammar
and then sitting down and computing the probabilities
of the rules manually. This is a very time-consuming
and unjustified enormous effort for documents of relatively
small size (let alone medium-size or million-word corpora).
Hence, there is a need for automatic tools and/or pre-existing
(and freely available!) treebanks to learn the grammars from.
For example, we have spent two days developing grammar for a one-sheet
document (see \xs{sect:realistic-grammar}) and the end result is that
we can only parse 3 (three) sentences so far with it.

\paragraph*{Chomsky Normal Form}

Another problem is the conversion of existing grammars
or maintaining the current grammar to make sure it is in the CNF
as the CYK algorithm \cite{cyk} requires. This is a problem because
for a human maintainer the number of rules grows, so it becomes
less obvious what a initial rule was like, and there's always
a possibility to create more undesired parses that way. We had
to do that for the Extended and Realistic Grammars that were
based on the grammar rules from the book \cite{jurafsky}.

To illustrate the problem, the rules similar to the
below have to be converted into CNF:

\begin{verbatim}
<A> -> <B>
<A> -> t <C>
<A> -> <B> <C> <D>
<A> -> <B> <C> t <D> <E>
\end{verbatim}

The conversion implies creating new non-terminals augmenting
the number of rules, which may be hard to trace later on when
there are many.

\begin{verbatim}
<A>   -> <B> <A>
<A>   -> <T> <C>
<A>   -> <BC> <D>
<A>   -> <BCT> <DE>
<T>   -> t
<BC>  -> <B> <C>
<DE>  -> <D> <DE>
<BCT> -> <BC> <T>
\end{verbatim}

The rule-set has been doubled in the above example. That creates
a problem of distributing the probability to the new rules as
well as the problem below.

\paragraph*{Exponential Growth of Storage Requirements and Run Time}

While playing with the Basic and Extended grammars, we didn't pay
much attention to the run-time aspect of the algorithm (even though
the number of the nested for-loops looked alerting) because
it was a second or less for the sentence set in \xf{fig:test-sentences}.
It became a problem, however, when we came up with the Realistic
grammar. The run-time for this grammar has jumped to 16 (sixteen !!!)
{\bf minutes} (!!!) in average for the sentence set in \xf{fig:asmt-sentences}.
This was rather discouraging (unless the goal is not the speed but
the most correct result at the hopefully near end). The problem stems
from the algorithm's complexity and huge data sparseness for large grammars.
One of the major reasons of the data sparseness problem is the CNF requirement as
described above: the number of non-terminals grows rapidly largely
increasing one of the dimensions of our $\pi$ and \verb+back+ arrays causing
number of iteration of the parsing algorithm to increase exponentially (or it is
at least cubic). And a lot of time is spent to find out that there's no
a parse for a sentence (given all the words of the sentence in the grammar).

\paragraph*{Data Sparseness and Smoothing}

The bigger grammar is the more severe the data sparseness is in our arrays
in this algorithm causing the above problem of the run-time and storage
requirements. Smoothing techniques we previously implemented in \api{LangIdentApp}
can be applied to at least get rid of zero-probabilities in the $\pi$ array, but we believe
the smoothing might hurt rather than improve the parsing performance;
yet we haven't tested this claim out yet (a TODO). A better way could be smooth
the grammar rules' probabilities.


%%%%%%%%%%%%%%%%%%%%%%%%%%%%%%%%%%%%%%%%%%%%%%%%%%%%%
%%%%%%%%%%%%%%%%%%%%%%%%%%%%%%%%%%%%%%%%%%%%%%%%%%%%%

\subsubsection{Results and Conclusions}

\paragraph{Generalities}

The algorithm does indeed seem to work and accept
syntactically-valid sentences while rejecting the ungrammatical
ones. Though to exhaust all the possible grammatical
and ungrammatical sentences in testing would require a lot more time.
There is also some disappointment
with respect to the running time increase when
the grammar grows and the other problems mentioned before.

To overcome the run-time problem, we'd have to use some
more sophisticated data structures than a plain 3D array
of doubles, but this is like fighting with the disease, instead
of its cause. The CYK algorithm has to be optimized
or even re-born to allow more than just CNF grammars
and be faster at the same time.

The restrictions on the sentences mentioned earlier can all
be overcome by just only tweaking the grammar (but increasing
the run-time along the way).

While most of the results of my test runs are in the further sections,
below we present one interactive session sample as well
as our favorite parse.

\paragraph{Sample Interactive Run}

\scriptsize
\input{prob-app-sample-run}
\normalsize


\paragraph{Favorite Parse}

As of this release, we were able to only parse the three sentences
from the ``requirements corpus'' and below is the favorite:

\begin{verbatim}
<S> (7.381879524149979E-10) [ 0-7: describe your grammar and how you developed it ]
    <S> (4.11319751814313E-4) [ 0-2: describe your grammar ]
        <V> (0.206897) [ 0-0: describe ]
        <NOMINAL> (0.039760823193600005) [ 1-2: your grammar ]
            <ADJ> (0.439024) [ 1-1: your ]
            <NOMINAL> (0.113208) [ 2-2: grammar ]
    <ConjS> (1.7946815078995936E-5) [ 3-7: and how you developed it ]
        <Conj> (0.92) [ 3-3: and ]
        <S> (1.9507407694560798E-5) [ 4-7: how you developed it ]
            <WhNP> (0.094285785) [ 4-5: how you ]
                <WhWord> (0.5) [ 4-4: how ]
                <Pronoun> (0.571429) [ 5-5: you ]
            <VP> (0.002068965931032) [ 6-7: developed it ]
                <V> (0.0344828) [ 6-6: developed ]
                <Pronoun> (0.333333) [ 7-7: it ]
\end{verbatim}


\subsubsection{Mini User Manual}
\label{sect:prob-manual}

\paragraph{System Requirements}

The program was mostly developed under Linux, so there's
a \file{Makefile} and a testing shell script to simplify some routine tasks.
For JVM, any JDK 1.4.* and above will do. \tool{bash} would be nice
to have to be able to run the batch script. Since the application
itself is written in {\java}, it's not bound to a specific architecture,
thus may be compiled and run without the makefiles and scripts
on virtually any operating system.

\paragraph{How To Run It}

There are thousands of ways how to run the program. Some of them are listed below.

\paragraph*{Using the Shell Script}

There is a script out there -- \tool{testing.sh}. It simply does compilation
and batch processing for all the three grammars and two sets of test
sentences in one run.
The script is written using \tool{bash} syntax; hence, \tool{bash} should be
present.

Type:

\noindent
\verb+./testing.sh+

or

\noindent
\verb+time ./testing.sh+

to execute  the batch (and time it in the second case). And example
of what one can get is below. NOTE: processing the first grammar, in
\file{grammar.asmt.txt}, may take awhile (below it took us 16 minutes),
so be aware of that fact (see the reasons in \xs{sect:prob-app-troubles}).

E.g.:

\begin{verbatim}
junebug.mokhov [a3] % time ./testing.sh
Making sure java files get compiled...
javac -g ProbabilisticParsingApp.java
Testing...
Compiling grammar: grammar.asmt.txt
Parsing...991.318u 1.511s 16:35.55 99.7%        0+0k 0+0io 5638pf+0w
Done
Look for parsing results in grammar.asmt.txt-parse.log

Compiling grammar: grammar.extended.txt
Parsing...1.591u 0.062s 0:01.82 90.6%   0+0k 0+0io 5637pf+0w
Done
Look for parsing results in grammar.extended.txt-parse.log

Compiling grammar: grammar.original.txt
Parsing...0.455u 0.066s 0:00.72 70.8%   0+0k 0+0io 5599pf+0w
Done
Look for parsing results in grammar.original.txt-parse.log

Testing done.
995.675u 1.906s 16:41.68 99.5%  0+0k 0+0io 42211pf+0w
junebug.mokhov [a3] %
\end{verbatim}

\paragraph*{Running ProbabilisitcParsingApp}
\label{sect:prob-parsing-app}

You can run the application itself without any wrapping scripts
and provide options to it. This is a command-line application,
so there is no GUI associated with it yet. To run the application
you have to compile it first. You can use either \tool{make} with no
arguments to compile or use a standard Java compiler.

Type:

\noindent
\verb+make+

or

\noindent
\verb+javac -cp marf.jar:. ProbabilisitcParsingApp.java+

After having compiled the application, you can run it with a JVM.
There are mutually-exclusive required options:

\begin{itemize}

\item
\verb+--train <grammar-file>+ -- to compile a grammar from the specified text file.
This is the first thing you need to do before trying to parse any sentences. Once compiled,
you don't need to recompile it each time you run the parser unless you made a fresh
copy of the application or made changes to the grammar file or plan to use a grammar
from another file.

\item
\verb+--parse+ -- to actually run the parser in the interactive mode (or batch mode, just use
input file redirection with your test sentences). To run the parser successfully there should
be compiled grammar first (see \verb+--train+).

\end{itemize}

E.g.:

To compile the Basic Grammar:

\verb+java -cp marf.jar:. ProbabilisitcParsingApp.java --train grammars/grammar.original.txt+

To batch process the sentence set from a file:

\verb+java -cp marf.jar:. ProbabilisitcParsingApp.java --parse < data/test-sentences.txt+

To run the application interactively:

\verb+java -cp marf.jar:. ProbabilisitcParsingApp.java --parse+

\noindent
Complete usage information:

\vspace{15pt}
\hrule
\begin{verbatim}

Probabilistic Parsing
Serguei A. Mokhov, mokhov@cs.concordia.ca
April 2003 - 2006


Usage:
    java ProbabilisticParsingApp --help | -h
        : to display this help and exit

    java ProbabilisticParsingApp --version
        : to display version and exit

    java ProbabilisticParsingApp --train [ OPTIONS ] <grammar-file>
        : to compile grammar from the <grammar-file>

    java ProbabilisticParsingApp --parse [ OPTIONS ]
        : to parse sentences from standard input

Where options are of the following:

    --debug  - enable debugging (more verbose output)
    -case    - make it case-sensitive
    -num     - parse numerical values
    -quote   - consider quotes and count quoted strings as one token
    -eos     - make typical ends of sentences (<?>, <!>, <.>) significant


\end{verbatim}

\hrule
\vspace{15pt}


\paragraph{List of Files of Interest}

\paragraph*{Directories}

\begin{itemize}

\item
\file{marf/nlp/} -- that's where most of the code is for this application is, the \api{marf.nlp} package.

\item
\file{marf/nlp/Parsing/} -- that's where most of the Parsing code is for Probabilistic Grammars and
Grammars in general

\item
\file{marf/nlp/Parsing/GrammarCompiler/} -- that's where the Grammar Compiler modules are

\end{itemize}

\paragraph*{The Application}

The application, its makefile, and
the wrapper script for batch training and testing.

\begin{verbatim}
ProbabilisticParsingApp.java
Makefile
testing.sh
marf.jar
\end{verbatim}

\paragraph*{Grammars}

\noindent
\file{grammars/grammar.original.txt} -- The Basic Grammar

\noindent
\file{grammars/grammar.extended.txt} -- The Extended Grammar

\noindent
\file{grammars/grammar.asmt.txt} -- The Realistic Grammar


\paragraph*{Test Sentences}

\noindent
\file{data/test-sentences.txt} -- the sample sentences from all over the place.

\noindent
\file{data/asmt-sentences.txt} -- the sentences derived from the requirements sheet.


%%%%%%%%%%%%%%%%%%%%%%%%%%%%%%%%%%%%%%%%%%%%%%%%%%%%
%%%%%%%%%%%%%%%%%%%%%%%%%%%%%%%%%%%%%%%%%%%%%%%%%%%%

\subsubsection{Results}
\label{sect:prob-parse-results}

\paragraph{Basic Grammar}

\tiny
\hrule\vskip4pt
\begin{verbatim}

Probabilistic Parsing
Serguei A. Mokhov, mokhov@cs
April 2003



Entering interactive mode... Type \q to exit.
sentence> my rabbit has a white smile

Parse for the sentence [ my rabbit has a white smile ] is below:

SYNOPSIS:

<NONTERMINAL> (PROBABILITY) [ SPAN: words of span ]

<S> (0.0020480000000000008) [ 0-5: my rabbit has a white smile ]
    <NP> (0.04000000000000001) [ 0-1: my rabbit ]
        <DET> (0.1) [ 0-0: my ]
        <NOMINAL> (0.4) [ 1-1: rabbit ]
    <VP> (0.06400000000000002) [ 2-5: has a white smile ]
        <V> (0.8) [ 2-2: has ]
        <NP> (0.08000000000000002) [ 3-5: a white smile ]
            <DET> (0.4) [ 3-3: a ]
            <NOMINAL> (0.2) [ 4-5: white smile ]
                <ADJ> (1.0) [ 4-4: white ]
                <NOMINAL> (0.2) [ 5-5: smile ]


sentence> my rabbit has a smile

Parse for the sentence [ my rabbit has a smile ] is below:

SYNOPSIS:

<NONTERMINAL> (PROBABILITY) [ SPAN: words of span ]

<S> (0.0020480000000000008) [ 0-4: my rabbit has a smile ]
    <NP> (0.04000000000000001) [ 0-1: my rabbit ]
        <DET> (0.1) [ 0-0: my ]
        <NOMINAL> (0.4) [ 1-1: rabbit ]
    <VP> (0.06400000000000002) [ 2-4: has a smile ]
        <V> (0.8) [ 2-2: has ]
        <NP> (0.08000000000000002) [ 3-4: a smile ]
            <DET> (0.4) [ 3-3: a ]
            <NOMINAL> (0.2) [ 4-4: smile ]


sentence> my rabbit has a telephone

There's no parse for [ my rabbit has a telephone ]

sentence> rabbit my a white has smile

There's no parse for [ rabbit my a white has smile ]

sentence> a slim blue refrigerator jumped gracefully out of the bottle

There's no parse for [ a slim blue refrigerator jumped gracefully out of the bottle ]

sentence> my smile has a rabbit

Parse for the sentence [ my smile has a rabbit ] is below:

SYNOPSIS:

<NONTERMINAL> (PROBABILITY) [ SPAN: words of span ]

<S> (0.0020480000000000008) [ 0-4: my smile has a rabbit ]
    <NP> (0.020000000000000004) [ 0-1: my smile ]
        <DET> (0.1) [ 0-0: my ]
        <NOMINAL> (0.2) [ 1-1: smile ]
    <VP> (0.12800000000000003) [ 2-4: has a rabbit ]
        <V> (0.8) [ 2-2: has ]
        <NP> (0.16000000000000003) [ 3-4: a rabbit ]
            <DET> (0.4) [ 3-3: a ]
            <NOMINAL> (0.4) [ 4-4: rabbit ]


sentence> the cat eats my white rabbit

Parse for the sentence [ the cat eats my white rabbit ] is below:

SYNOPSIS:

<NONTERMINAL> (PROBABILITY) [ SPAN: words of span ]

<S> (0.0012800000000000005) [ 0-5: the cat eats my white rabbit ]
    <NP> (0.2) [ 0-1: the cat ]
        <DET> (0.5) [ 0-0: the ]
        <NOMINAL> (0.4) [ 1-1: cat ]
    <VP> (0.008000000000000002) [ 2-5: eats my white rabbit ]
        <V> (0.2) [ 2-2: eats ]
        <NP> (0.04000000000000001) [ 3-5: my white rabbit ]
            <DET> (0.1) [ 3-3: my ]
            <NOMINAL> (0.4) [ 4-5: white rabbit ]
                <ADJ> (1.0) [ 4-4: white ]
                <NOMINAL> (0.4) [ 5-5: rabbit ]


sentence> a white smile eats the cat

Parse for the sentence [ a white smile eats the cat ] is below:

SYNOPSIS:

<NONTERMINAL> (PROBABILITY) [ SPAN: words of span ]

<S> (0.0025600000000000015) [ 0-5: a white smile eats the cat ]
    <NP> (0.08000000000000002) [ 0-2: a white smile ]
        <DET> (0.4) [ 0-0: a ]
        <NOMINAL> (0.2) [ 1-2: white smile ]
            <ADJ> (1.0) [ 1-1: white ]
            <NOMINAL> (0.2) [ 2-2: smile ]
    <VP> (0.04000000000000001) [ 3-5: eats the cat ]
        <V> (0.2) [ 3-3: eats ]
        <NP> (0.2) [ 4-5: the cat ]
            <DET> (0.5) [ 4-4: the ]
            <NOMINAL> (0.4) [ 5-5: cat ]


sentence> my cat has a white rabbit

Parse for the sentence [ my cat has a white rabbit ] is below:

SYNOPSIS:

<NONTERMINAL> (PROBABILITY) [ SPAN: words of span ]

<S> (0.0040960000000000015) [ 0-5: my cat has a white rabbit ]
    <NP> (0.04000000000000001) [ 0-1: my cat ]
        <DET> (0.1) [ 0-0: my ]
        <NOMINAL> (0.4) [ 1-1: cat ]
    <VP> (0.12800000000000003) [ 2-5: has a white rabbit ]
        <V> (0.8) [ 2-2: has ]
        <NP> (0.16000000000000003) [ 3-5: a white rabbit ]
            <DET> (0.4) [ 3-3: a ]
            <NOMINAL> (0.4) [ 4-5: white rabbit ]
                <ADJ> (1.0) [ 4-4: white ]
                <NOMINAL> (0.4) [ 5-5: rabbit ]


sentence> my cat has white rabbit

There's no parse for [ my cat has white rabbit ]

sentence> cat has a white rabbit

There's no parse for [ cat has a white rabbit ]

sentence> smile has my cat

There's no parse for [ smile has my cat ]

sentence> can you book TWA flights

There's no parse for [ can you book TWA flights ]

sentence> the lion jumped through the hoop

There's no parse for [ the lion jumped through the hoop ]

sentence> the trainer jumped the lion through the hoop

There's no parse for [ the trainer jumped the lion through the hoop ]

sentence> the butter melted in the pan

There's no parse for [ the butter melted in the pan ]

sentence> the cook melted the butter in the pan

There's no parse for [ the cook melted the butter in the pan ]

sentence> the rich love their money

There's no parse for [ the rich love their money ]

sentence> the rich love sometimes too

There's no parse for [ the rich love sometimes too ]

sentence> the contractor built the houses last summer

There's no parse for [ the contractor built the houses last summer ]

sentence> the contractor built last summer

There's no parse for [ the contractor built last summer ]

sentence>
\end{verbatim}
\vskip4pt\hrule

\paragraph{Extended Grammar}

\tiny
\hrule\vskip4pt
\begin{verbatim}

Probabilistic Parsing
Serguei A. Mokhov, mokhov@cs
April 2003



Entering interactive mode... Type \q to exit.
sentence> my rabbit has a white smile

Parse for the sentence [ my rabbit has a white smile ] is below:

SYNOPSIS:

<NONTERMINAL> (PROBABILITY) [ SPAN: words of span ]

<S> (1.915200000000001E-6) [ 0-5: my rabbit has a white smile ]
    <NP> (0.0020000000000000005) [ 0-1: my rabbit ]
        <DET> (0.05) [ 0-0: my ]
        <NOMINAL> (0.2) [ 1-1: rabbit ]
    <VP> (0.002736000000000001) [ 2-5: has a white smile ]
        <V> (0.5) [ 2-2: has ]
        <NP> (0.005760000000000002) [ 3-5: a white smile ]
            <DET> (0.4) [ 3-3: a ]
            <NOMINAL> (0.07200000000000002) [ 4-5: white smile ]
                <ADJ> (0.8) [ 4-4: white ]
                <NOMINAL> (0.1) [ 5-5: smile ]


sentence> my rabbit has a smile

Parse for the sentence [ my rabbit has a smile ] is below:

SYNOPSIS:

<NONTERMINAL> (PROBABILITY) [ SPAN: words of span ]

<S> (2.6600000000000012E-6) [ 0-4: my rabbit has a smile ]
    <NP> (0.0020000000000000005) [ 0-1: my rabbit ]
        <DET> (0.05) [ 0-0: my ]
        <NOMINAL> (0.2) [ 1-1: rabbit ]
    <VP> (0.003800000000000001) [ 2-4: has a smile ]
        <V> (0.5) [ 2-2: has ]
        <NP> (0.008000000000000002) [ 3-4: a smile ]
            <DET> (0.4) [ 3-3: a ]
            <NOMINAL> (0.1) [ 4-4: smile ]


sentence> my rabbit has a telephone

There's no parse for [ my rabbit has a telephone ]

sentence> rabbit my a white has smile

There's no parse for [ rabbit my a white has smile ]

sentence> a slim blue refrigerator jumped gracefully out of the bottle

There's no parse for [ a slim blue refrigerator jumped gracefully out of the bottle ]

sentence> my smile has a rabbit

Parse for the sentence [ my smile has a rabbit ] is below:

SYNOPSIS:

<NONTERMINAL> (PROBABILITY) [ SPAN: words of span ]

<S> (2.6600000000000012E-6) [ 0-4: my smile has a rabbit ]
    <NP> (0.0010000000000000002) [ 0-1: my smile ]
        <DET> (0.05) [ 0-0: my ]
        <NOMINAL> (0.1) [ 1-1: smile ]
    <VP> (0.007600000000000002) [ 2-4: has a rabbit ]
        <V> (0.5) [ 2-2: has ]
        <NP> (0.016000000000000004) [ 3-4: a rabbit ]
            <DET> (0.4) [ 3-3: a ]
            <NOMINAL> (0.2) [ 4-4: rabbit ]


sentence> the cat eats my white rabbit

Parse for the sentence [ the cat eats my white rabbit ] is below:

SYNOPSIS:

<NONTERMINAL> (PROBABILITY) [ SPAN: words of span ]

<S> (9.576000000000004E-7) [ 0-5: the cat eats my white rabbit ]
    <NP> (0.020000000000000004) [ 0-1: the cat ]
        <DET> (0.5) [ 0-0: the ]
        <NOMINAL> (0.2) [ 1-1: cat ]
    <VP> (1.3680000000000005E-4) [ 2-5: eats my white rabbit ]
        <V> (0.1) [ 2-2: eats ]
        <NP> (0.0014400000000000005) [ 3-5: my white rabbit ]
            <DET> (0.05) [ 3-3: my ]
            <NOMINAL> (0.14400000000000004) [ 4-5: white rabbit ]
                <ADJ> (0.8) [ 4-4: white ]
                <NOMINAL> (0.2) [ 5-5: rabbit ]


sentence> a white smile eats the cat

Parse for the sentence [ a white smile eats the cat ] is below:

SYNOPSIS:

<NONTERMINAL> (PROBABILITY) [ SPAN: words of span ]

<S> (3.8304000000000015E-6) [ 0-5: a white smile eats the cat ]
    <NP> (0.005760000000000002) [ 0-2: a white smile ]
        <DET> (0.4) [ 0-0: a ]
        <NOMINAL> (0.07200000000000002) [ 1-2: white smile ]
            <ADJ> (0.8) [ 1-1: white ]
            <NOMINAL> (0.1) [ 2-2: smile ]
    <VP> (0.0019000000000000004) [ 3-5: eats the cat ]
        <V> (0.1) [ 3-3: eats ]
        <NP> (0.020000000000000004) [ 4-5: the cat ]
            <DET> (0.5) [ 4-4: the ]
            <NOMINAL> (0.2) [ 5-5: cat ]


sentence> my cat has a white rabbit

Parse for the sentence [ my cat has a white rabbit ] is below:

SYNOPSIS:

<NONTERMINAL> (PROBABILITY) [ SPAN: words of span ]

<S> (3.830400000000002E-6) [ 0-5: my cat has a white rabbit ]
    <NP> (0.0020000000000000005) [ 0-1: my cat ]
        <DET> (0.05) [ 0-0: my ]
        <NOMINAL> (0.2) [ 1-1: cat ]
    <VP> (0.005472000000000002) [ 2-5: has a white rabbit ]
        <V> (0.5) [ 2-2: has ]
        <NP> (0.011520000000000004) [ 3-5: a white rabbit ]
            <DET> (0.4) [ 3-3: a ]
            <NOMINAL> (0.14400000000000004) [ 4-5: white rabbit ]
                <ADJ> (0.8) [ 4-4: white ]
                <NOMINAL> (0.2) [ 5-5: rabbit ]


sentence> my cat has white rabbit

There's no parse for [ my cat has white rabbit ]

sentence> cat has a white rabbit

Parse for the sentence [ cat has a white rabbit ] is below:

SYNOPSIS:

<NONTERMINAL> (PROBABILITY) [ SPAN: words of span ]

<S> (3.283200000000001E-5) [ 0-4: cat has a white rabbit ]
    <NOMINAL> (0.2) [ 0-0: cat ]
    <VP> (0.005472000000000002) [ 1-4: has a white rabbit ]
        <V> (0.5) [ 1-1: has ]
        <NP> (0.011520000000000004) [ 2-4: a white rabbit ]
            <DET> (0.4) [ 2-2: a ]
            <NOMINAL> (0.14400000000000004) [ 3-4: white rabbit ]
                <ADJ> (0.8) [ 3-3: white ]
                <NOMINAL> (0.2) [ 4-4: rabbit ]


sentence> smile has my cat

Parse for the sentence [ smile has my cat ] is below:

SYNOPSIS:

<NONTERMINAL> (PROBABILITY) [ SPAN: words of span ]

<S> (2.8500000000000007E-6) [ 0-3: smile has my cat ]
    <NOMINAL> (0.1) [ 0-0: smile ]
    <VP> (9.500000000000002E-4) [ 1-3: has my cat ]
        <V> (0.5) [ 1-1: has ]
        <NP> (0.0020000000000000005) [ 2-3: my cat ]
            <DET> (0.05) [ 2-2: my ]
            <NOMINAL> (0.2) [ 3-3: cat ]


sentence> can you book TWA flights

Parse for the sentence [ can you book TWA flights ] is below:

SYNOPSIS:

<NONTERMINAL> (PROBABILITY) [ SPAN: words of span ]

<S> (3.192E-5) [ 0-4: can you book TWA flights ]
    <AuxNP> (0.09600000000000002) [ 0-1: can you ]
        <Aux> (0.4) [ 0-0: can ]
        <Pronoun> (0.4) [ 1-1: you ]
    <VP> (0.0033249999999999994) [ 2-4: book TWA flights ]
        <V> (0.1) [ 2-2: book ]
        <NP> (0.034999999999999996) [ 3-4: TWA flights ]
            <ProperNoun> (0.4) [ 3-3: TWA ]
            <NOMINAL> (0.25) [ 4-4: flights ]


sentence> the lion jumped through the hoop

There's no parse for [ the lion jumped through the hoop ]

sentence> the trainer jumped the lion through the hoop

There's no parse for [ the trainer jumped the lion through the hoop ]

sentence> the butter melted in the pan

There's no parse for [ the butter melted in the pan ]

sentence> the cook melted the butter in the pan

There's no parse for [ the cook melted the butter in the pan ]

sentence> the rich love their money

There's no parse for [ the rich love their money ]

sentence> the rich love sometimes too

There's no parse for [ the rich love sometimes too ]

sentence> the contractor built the houses last summer

There's no parse for [ the contractor built the houses last summer ]

sentence> the contractor built last summer

There's no parse for [ the contractor built last summer ]

sentence>
\end{verbatim}
\vskip4pt\hrule

\paragraph{Realistic Grammar}

\tiny
\hrule\vskip4pt
\begin{verbatim}

Probabilistic Parsing
Serguei A. Mokhov, mokhov@cs
April 2003



Entering interactive mode... Type \q to exit.
sentence> you should submit a paper listing and report and an electronic version of your code

There's no parse for [ you should submit a paper listing and report and an electronic version of your code ]

sentence> implement the CYK algorithm to find the best parse for a given sentence

There's no parse for [ implement the CYK algorithm to find the best parse for a given sentence ]

sentence> your program should take as input a probabilistic grammar and a sentence and display the best parse tree along with its probability

There's no parse for [ your program should take as input a probabilistic grammar and a sentence and display the best parse tree along with its probability ]

sentence> you are not required to use a specific programming language

There's no parse for [ you are not required to use a specific programming language ]

sentence> Perl, CPP, C or Java are appropriate for this task
WARNING: Non-word token encountered: Token[','], line 1
WARNING: Non-word token encountered: Token[','], line 1

There's no parse for [ Perl, CPP, C or Java are appropriate for this task ]

sentence> as long as you explain it, your grammar can be in any format you wish
WARNING: Non-word token encountered: Token[','], line 1

There's no parse for [ as long as you explain it, your grammar can be in any format you wish ]

sentence> your grammar can be in any format you wish as long as you explain it

There's no parse for [ your grammar can be in any format you wish as long as you explain it ]

sentence> experiment with your grammar

There's no parse for [ experiment with your grammar ]

sentence> is it restrictive enough

There's no parse for [ is it restrictive enough ]

sentence> does it refuse ungrammatical sentences

Parse for the sentence [ does it refuse ungrammatical sentences ] is below:

SYNOPSIS:

<NONTERMINAL> (PROBABILITY) [ SPAN: words of span ]

<S> (1.294887213288665E-8) [ 0-4: does it refuse ungrammatical sentences ]
    <AuxNP> (0.01999998) [ 0-1: does it ]
        <Aux> (0.1) [ 0-0: does ]
        <Pronoun> (0.333333) [ 1-1: it ]
    <VP> (6.474442540885865E-6) [ 2-4: refuse ungrammatical sentences ]
        <V> (0.0344828) [ 2-2: refuse ]
        <NOMINAL> (0.001104462402208) [ 3-4: ungrammatical sentences ]
            <ADJ> (0.0243902) [ 3-3: ungrammatical ]
            <NOMINAL> (0.0566038) [ 4-4: sentences ]


sentence> does it cover all grammatical sentences

Parse for the sentence [ does it cover all grammatical sentences ] is below:

SYNOPSIS:

<NONTERMINAL> (PROBABILITY) [ SPAN: words of span ]

<S> (1.2948872132886648E-9) [ 0-5: does it cover all grammatical sentences ]
    <AuxNP> (0.01999998) [ 0-1: does it ]
        <Aux> (0.1) [ 0-0: does ]
        <Pronoun> (0.333333) [ 1-1: it ]
    <VP> (6.474442540885865E-7) [ 2-5: cover all grammatical sentences ]
        <V> (0.0344828) [ 2-2: cover ]
        <NP> (5.52231201104E-5) [ 3-5: all grammatical sentences ]
            <PreDet> (0.5) [ 3-3: all ]
            <NOMINAL> (0.001104462402208) [ 4-5: grammatical sentences ]
                <ADJ> (0.0243902) [ 4-4: grammatical ]
                <NOMINAL> (0.0566038) [ 5-5: sentences ]


sentence> write a report to describe your code and your experimentation

There's no parse for [ write a report to describe your code and your experimentation ]

sentence> your report must describe the program

There's no parse for [ your report must describe the program ]

sentence> your report must describe the experiments

There's no parse for [ your report must describe the experiments ]

sentence> describe your code itself

There's no parse for [ describe your code itself ]

sentence> indicate the instructions necessary to run your code

There's no parse for [ indicate the instructions necessary to run your code ]

sentence> describe your choice of test sentences

There's no parse for [ describe your choice of test sentences ]

sentence> describe your grammar and how you developed it

Parse for the sentence [ describe your grammar and how you developed it ] is below:

SYNOPSIS:

<NONTERMINAL> (PROBABILITY) [ SPAN: words of span ]

<S> (7.381879524149979E-10) [ 0-7: describe your grammar and how you developed it ]
    <S> (4.11319751814313E-4) [ 0-2: describe your grammar ]
        <V> (0.206897) [ 0-0: describe ]
        <NOMINAL> (0.039760823193600005) [ 1-2: your grammar ]
            <ADJ> (0.439024) [ 1-1: your ]
            <NOMINAL> (0.113208) [ 2-2: grammar ]
    <ConjS> (1.7946815078995936E-5) [ 3-7: and how you developed it ]
        <Conj> (0.92) [ 3-3: and ]
        <S> (1.9507407694560798E-5) [ 4-7: how you developed it ]
            <WhNP> (0.094285785) [ 4-5: how you ]
                <WhWord> (0.5) [ 4-4: how ]
                <Pronoun> (0.571429) [ 5-5: you ]
            <VP> (0.002068965931032) [ 6-7: developed it ]
                <V> (0.0344828) [ 6-6: developed ]
                <Pronoun> (0.333333) [ 7-7: it ]


sentence> what problems do you see with your current implementation

There's no parse for [ what problems do you see with your current implementation ]

sentence> what problems do you see with your current grammar

There's no parse for [ what problems do you see with your current grammar ]

sentence> how would you improve it

There's no parse for [ how would you improve it ]

sentence> both the code and the report must be submitted electronically and in paper

There's no parse for [ both the code and the report must be submitted electronically and in paper ]

sentence> in class, you must submit a listing of your program and results and the report
WARNING: Non-word token encountered: Token[','], line 1

There's no parse for [ in class, you must submit a listing of your program and results and the report ]

sentence> through the electronic submission form you must submit the code of your program and results and an electronic version of your report

There's no parse for [ through the electronic submission form you must submit the code of your program and results and an electronic version of your report ]

sentence>
\end{verbatim}
\vskip4pt\hrule

\normalsize


\subsubsection{Code}
\label{sect:prob-parse-code}

The entire source code is provided in the electronic form, but
here we provide the two methods extracted from
\file{marf/nlp/Parsing/ProbabilisticParser.java} that are actual
implementation of the CYK algorithm and the \verb+build_tree()+ function,
named \api{parse()} and \file{dumpParseTree()} respectively.
The code below has experienced minor clean-ups in the report
version with most of the debug and error handling information removed. For the
unaltered code please refer the above mentioned file itself.

\paragraph{\texttt{ProbabilisticParser.parse() -- CYK}}

\tiny
\begin{verbatim}
public boolean parse()
throws SyntaxError
{
    // Restore grammar from the disk
    restore();

    // Split the string into words

    this.oWords = new Vector();

    int iTokenType;

    while((iTokenType = this.oStreamTokenizer.nextToken()) != StreamTokenizer.TT_EOF)
    {
        switch(iTokenType)
        {
            case StreamTokenizer.TT_WORD:
            {
                this.oWords.add(new String(this.oStreamTokenizer.sval));
                break;
            }

            case StreamTokenizer.TT_NUMBER:
            default:
            {
                System.err.println("WARNING: Non-word token encountered: " + this.oStreamTokenizer);
                break;
            }
        }
    }

    // CYK
    Vector oNonTerminals = this.oGrammar.getNonTerminalList();

    this.adParseMatrix = new double[this.oWords.size()][this.oWords.size()][oNonTerminals.size()];
    this.aoBack        = new Vector[this.oWords.size()][this.oWords.size()][oNonTerminals.size()];

    // Base case

    for(int i = 0; i < this.oWords.size(); i++)
    {
        String strTerminal = this.oWords.elementAt(i).toString();

        /*
         * Fail-fast: if the terminal is not in the grammar (no rule 'A->wd' found ),
         * there is no point to compute the parse
         */
        if(this.oGrammar.containsTerminal(strTerminal) == -1)
        {
            return false;
        }

        for(int A = 0; A < oNonTerminals.size(); A++)
        {
            ProbabilisticRule oRule = (ProbabilisticRule)this.oGrammar.getRule(strTerminal, A);

            // Current pair: 'A->wd' does not form a Rule
            if(oRule == null)
            {
                continue;
            }

            this.adParseMatrix[i][i][A] = oRule.getProbability();
        }
    }

    /*
     * Recursive case
     * ('recursive' as authors call it, but it's implemented iteratively
     * and me being just a copy-cat here)
     */
    for(int iSpan = 2; iSpan <= this.oWords.size(); iSpan++)
    {
        for(int iBegin = 0; iBegin < this.oWords.size() - iSpan + 1; iBegin++)
        {
            int iEnd = iBegin + iSpan - 1;

            // For every split m of the incoming sentence ...
            for(int m = iBegin; m <= iEnd - 1; m++)
            {
                // Check how the split divides B and C in A->BC
                for(int iA = 0; iA < oNonTerminals.size(); iA++)
                {
                    for(int iB = 0; iB < oNonTerminals.size(); iB++)
                    {
                        for(int iC = 0; iC < oNonTerminals.size(); iC++)
                        {
                            ProbabilisticRule oRule = (ProbabilisticRule)this.oGrammar.getRule(iA, iB, iC);

                            if(oRule == null)
                            {
                                continue;
                            }

                            double dProb =
                                this.adParseMatrix[iBegin][m][iB] *
                                this.adParseMatrix[m + 1][iEnd][iC] *
                                oRule.getProbability();

                            if(dProb > this.adParseMatrix[iBegin][iEnd][iA])
                            {
                                this.adParseMatrix[iBegin][iEnd][iA] = dProb;

                                Vector oBackIndices = new Vector();

                                oBackIndices.add(new Integer(m));
                                oBackIndices.add(new Integer(iB));
                                oBackIndices.add(new Integer(iC));

                                this.aoBack[iBegin][iEnd][iA] = oBackIndices;
                            }
                        } // for C
                    } // for B
                } // for A
            } // split
        }
    } // "recursive" case

    // No parse
    if(this.adParseMatrix[0][this.oWords.size() - 1][0] == 0)
    {
        return false;
    }

    return true;
}
\end{verbatim}
\normalsize

\paragraph{\texttt{ProbabilisticParser.dmpParseTree() -- build\_tree()}}

\tiny
\begin{verbatim}
public void dumpParseTree(int piLevel, int i, int j, int piA)
{
    NonTerminal oLHS = (NonTerminal)this.oGrammar.getNonTerminalList().elementAt(piA);

    indent(piLevel);

    // Termination case

    if(this.aoBack[i][j][piA] == null)
    {
        return;
    }

    // Recursive case

    int m = ((Integer)this.aoBack[i][j][piA].elementAt(0)).intValue();
    int iB = ((Integer)this.aoBack[i][j][piA].elementAt(1)).intValue();
    int iC = ((Integer)this.aoBack[i][j][piA].elementAt(2)).intValue();

    dumpParseTree(piLevel + 1, i, m, iB);
    dumpParseTree(piLevel + 1,  m + 1, j, iC);
}
\end{verbatim}
\normalsize

% EOF
