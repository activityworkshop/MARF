\subsection{Noise Removal}

Any vocal sample taken in a less-than-perfect (which is always the case) environment will experience a certain
amount of room noise.  Since background noise exhibits a certain frequency characteristic,
if the noise is loud enough it may inhibit good recognition of a voice when the voice is
later tested in a different environment.  Therefore, it is necessary to remove as much
environmental interference as possible.

To remove room noise, it is first necessary to get a sample of the room noise by itself.
This sample, usually at least 30 seconds long, should provide the general frequency
characteristics of the noise when subjected to FFT analysis.  Using a technique similar
to overlap-add FFT filtering, room noise can then be removed from the vocal sample by
simply subtracting the noise's frequency characteristics from the vocal sample in question.

That is, if $S(x)$ is the sample, $N(x)$ is the noise, and $V(x)$ is the voice, all in the
frequency domain, then

\begin{center}$S(x) = N(x) + V(x)$\end{center}

Therefore, it should be possible to isolate the voice:

\begin{center}$V(x) = S(x) - N(x)$\end{center}

Unfortunately, time has not permitted us to implement this in practice yet.
