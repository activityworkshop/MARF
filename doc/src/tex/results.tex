\chapter{Experimentation Results}
\label{sect:results}
\index{Results}

$Revision: 1.22 $

\section{Notes}

Before we get to numbers, few notes and observations first:

\begin{enumerate}

\item We've got more samples since the demo. The obvious: by increasing the number of samples our results got
      better; with few exceptions, however. This can be explained by
      the diversity of the recording equipment, a lot less than uniform
      number of samples per speaker, and absence of noise and silence
      removal. All the samples were recorded in not the same environments.
      The results then start averaging after awhile.

\item Another observation we made from our output, is that
      when the speaker is guessed incorrectly, quite often the second
      guess is correct, so we included this in our results as if we were
      ``guessing'' right from the second attempt.

\item FUN. Interesting to note, that we also tried to take some
      samples of music bands, and feed it to our application
      along with the speakers, and application's performance didn't suffer,
      yet even improved because the samples were treated in
      the same manner. The groups were not mentioned in the table,
      so we name them here: Van Halen [8:1] and Red Hot Chili Peppers [10:1] (where numbers
      represent [training:testing] samples used).

\end{enumerate}

\clearpage

\section{\api{SpeakerIdentApp}'s Options}

Configuration parameters were extracted from the command line,
which \api{SpeakerIdentApp} can be invoked with. They mean the following:

\vspace{15pt}
\hrule
\scriptsize
\begin{verbatim}
Usage:
  java SpeakerIdentApp --train <samples-dir> [options]        -- train mode
                       --single-train <sample> [options]      -- add a single sample to the training set
                       --ident <sample> [options]             -- identification mode
                       --batch-ident <samples-dir> [options]  -- batch identification mode
                       --stats                                -- display stats
                       --best-score                           -- display best classification result
                       --reset                                -- reset stats
                       --version                              -- display version info
                       --help | -h                            -- display this help and exit

Options (one or more of the following):

Preprocessing:

  -raw          - no preprocessing
  -norm         - use just normalization, no filtering
  -low          - use low-pass filter
  -high         - use high-pass filter
  -boost        - use high-frequency-boost preprocessor
  -band         - use band-pass filter
  -endp         - use endpointing

Feature Extraction:

  -lpc          - use LPC
  -fft          - use FFT
  -minmax       - use Min/Max Amplitudes
  -randfe       - use random feature extraction
  -aggr         - use aggregated FFT+LPC feature extraction

Classification:

  -nn           - use Neural Network
  -cheb         - use Chebyshev Distance
  -eucl         - use Euclidean Distance
  -mink         - use Minkowski Distance
  -diff         - use Diff-Distance
  -randcl       - use random classification

Misc:

  -debug        - include verbose debug output
  -spectrogram  - dump spectrogram image after feature extraction
  -graph        - dump wave graph before preprocessing and after feature extraction
  <integer>     - expected speaker ID


\end{verbatim}

\normalsize
\hrule
\vspace{15pt}

\clearpage

% NOTE: stats-date.tex, best-score.tex, and stats.tex are generated by
%       the SpeakerIdentApp after running testing.sh

\section{Consolidated Results}

Our ultimate results \footnote{authoritative as of Tue Jan 24 06:23:21 EST 2006
} for all configurations we can have
and samples we've got are below.
Looks like our best results are with
``-endp -lpc -cheb'',
``-raw -aggr -eucl'',
``-norm -aggr -diff'',
``-norm -aggr -cheb'',
``-raw -aggr -mah'',
``-raw -fft -mah'',
``-raw -fft -eucl'', and
``-norm -fft -diff''
with the top result being around \bestscore{\input{best-score}} and the second-best is
around \bestscore{\input{second-best-score}} (see \xt{tab:results1}).

\begin{table}
\begin{minipage}[b]{\textwidth}
\centering
\begin{tabular}{|c|c|l|c|c|r|} \hline
Guess & Run \# & Configuration & GOOD & BAD & Recognition Rate,\%\\ \hline\hline
1st & 1 & -endp -lpc -cheb  & 24 & 5 & 82.76\\ \hline
1st & 2 & -raw -aggr -eucl  & 22 & 7 & 75.86\\ \hline
1st & 3 & -norm -aggr -diff  & 22 & 7 & 75.86\\ \hline
1st & 4 & -norm -aggr -cheb  & 22 & 7 & 75.86\\ \hline
1st & 5 & -raw -aggr -mah  & 22 & 7 & 75.86\\ \hline
1st & 6 & -raw -fft -mah  & 22 & 7 & 75.86\\ \hline
1st & 7 & -raw -fft -eucl  & 22 & 7 & 75.86\\ \hline
1st & 8 & -norm -fft -diff  & 22 & 7 & 75.86\\ \hline
1st & 9 & -norm -fft -cheb  & 21 & 8 & 72.41\\ \hline
1st & 10 & -raw -aggr -cheb  & 21 & 8 & 72.41\\ \hline
1st & 11 & -endp -lpc -mah  & 21 & 8 & 72.41\\ \hline
1st & 12 & -endp -lpc -eucl  & 21 & 8 & 72.41\\ \hline
1st & 13 & -raw -fft -mink  & 21 & 8 & 72.41\\ \hline
1st & 14 & -norm -fft -mah  & 21 & 8 & 72.41\\ \hline
1st & 15 & -norm -fft -eucl  & 21 & 8 & 72.41\\ \hline
1st & 16 & -norm -aggr -eucl  & 20 & 9 & 68.97\\ \hline
1st & 17 & -low -aggr -diff  & 20 & 9 & 68.97\\ \hline
1st & 18 & -norm -aggr -mah  & 20 & 9 & 68.97\\ \hline
1st & 19 & -raw -fft -cheb  & 20 & 9 & 68.97\\ \hline
1st & 20 & -raw -aggr -mink  & 20 & 9 & 68.97\\ \hline
1st & 21 & -norm -aggr -mink  & 19 & 10 & 65.52\\ \hline
1st & 22 & -low -fft -cheb  & 19 & 10 & 65.52\\ \hline
1st & 23 & -raw -lpc -mink  & 19 & 10 & 65.52\\ \hline
1st & 24 & -raw -lpc -diff  & 19 & 10 & 65.52\\ \hline
1st & 25 & -raw -lpc -eucl  & 19 & 10 & 65.52\\ \hline
1st & 26 & -raw -lpc -mah  & 19 & 10 & 65.52\\ \hline
1st & 27 & -raw -lpc -cheb  & 19 & 10 & 65.52\\ \hline
1st & 28 & -low -aggr -eucl  & 19 & 10 & 65.52\\ \hline
1st & 29 & -norm -lpc -mah  & 19 & 10 & 65.52\\ \hline
\end{tabular}
\end{minipage}
\caption{Consolidated results, Part 1.}
\label{tab:results1}
\end{table}

\begin{table}
\begin{minipage}[b]{\textwidth}
\centering
\begin{tabular}{|c|c|l|c|c|r|} \hline
Guess & Run \# & Configuration & GOOD & BAD & Recognition Rate,\%\\ \hline\hline
1st & 30 & -norm -lpc -mink  & 19 & 10 & 65.52\\ \hline
1st & 31 & -norm -lpc -diff  & 19 & 10 & 65.52\\ \hline
1st & 32 & -norm -lpc -eucl  & 19 & 10 & 65.52\\ \hline
1st & 33 & -low -aggr -cheb  & 19 & 10 & 65.52\\ \hline
1st & 34 & -norm -lpc -cheb  & 19 & 10 & 65.52\\ \hline
1st & 35 & -low -aggr -mah  & 19 & 10 & 65.52\\ \hline
1st & 36 & -low -fft -mah  & 19 & 10 & 65.52\\ \hline
1st & 37 & -norm -fft -mink  & 19 & 10 & 65.52\\ \hline
1st & 38 & -low -fft -diff  & 19 & 10 & 65.52\\ \hline
1st & 39 & -low -fft -eucl  & 19 & 10 & 65.52\\ \hline
1st & 40 & -raw -aggr -diff  & 19 & 10 & 65.52\\ \hline
1st & 41 & -high -aggr -mink  & 18 & 11 & 62.07\\ \hline
1st & 42 & -high -aggr -eucl  & 18 & 11 & 62.07\\ \hline
1st & 43 & -endp -lpc -mink  & 18 & 11 & 62.07\\ \hline
1st & 44 & -high -aggr -mah  & 18 & 11 & 62.07\\ \hline
1st & 45 & -raw -fft -diff  & 18 & 11 & 62.07\\ \hline
1st & 46 & -high -aggr -cheb  & 17 & 12 & 58.62\\ \hline
1st & 47 & -low -aggr -mink  & 17 & 12 & 58.62\\ \hline
1st & 48 & -low -fft -mink  & 17 & 12 & 58.62\\ \hline
1st & 49 & -high -fft -cheb  & 16 & 13 & 55.17\\ \hline
1st & 50 & -high -fft -mah  & 16 & 13 & 55.17\\ \hline
1st & 51 & -low -lpc -cheb  & 16 & 13 & 55.17\\ \hline
1st & 52 & -high -fft -mink  & 16 & 13 & 55.17\\ \hline
1st & 53 & -high -fft -eucl  & 16 & 13 & 55.17\\ \hline
1st & 54 & -low -lpc -eucl  & 15 & 14 & 51.72\\ \hline
1st & 55 & -low -lpc -mah  & 15 & 14 & 51.72\\ \hline
1st & 56 & -low -lpc -mink  & 14 & 15 & 48.28\\ \hline
1st & 57 & -low -lpc -diff  & 14 & 15 & 48.28\\ \hline
1st & 58 & -high -lpc -cheb  & 14 & 15 & 48.28\\ \hline
1st & 59 & -raw -lpc -nn  & 14 & 15 & 48.28\\ \hline
\end{tabular}
\end{minipage}
\caption{Consolidated results, Part 2.}
\label{tab:results2}
\end{table}

\begin{table}
\begin{minipage}[b]{\textwidth}
\centering
\begin{tabular}{|c|c|l|c|c|r|} \hline
Guess & Run \# & Configuration & GOOD & BAD & Recognition Rate,\%\\ \hline\hline
1st & 60 & -band -aggr -diff  & 13 & 16 & 44.83\\ \hline
1st & 61 & -norm -lpc -nn  & 13 & 16 & 44.83\\ \hline
1st & 62 & -band -fft -diff  & 13 & 16 & 44.83\\ \hline
1st & 63 & -high -lpc -eucl  & 12 & 17 & 41.38\\ \hline
1st & 64 & -high -aggr -diff  & 12 & 17 & 41.38\\ \hline
1st & 65 & -endp -fft -diff  & 12 & 17 & 41.38\\ \hline
1st & 66 & -endp -fft -eucl  & 12 & 17 & 41.38\\ \hline
1st & 67 & -band -lpc -mink  & 12 & 17 & 41.38\\ \hline
1st & 68 & -band -lpc -mah  & 12 & 17 & 41.38\\ \hline
1st & 69 & -band -lpc -eucl  & 12 & 17 & 41.38\\ \hline
1st & 70 & -endp -fft -cheb  & 12 & 17 & 41.38\\ \hline
1st & 71 & -band -lpc -cheb  & 12 & 17 & 41.38\\ \hline
1st & 72 & -endp -fft -mah  & 12 & 17 & 41.38\\ \hline
1st & 73 & -high -lpc -mah  & 12 & 17 & 41.38\\ \hline
1st & 74 & -endp -aggr -diff  & 12 & 17 & 41.38\\ \hline
1st & 75 & -endp -aggr -eucl  & 12 & 17 & 41.38\\ \hline
1st & 76 & -endp -aggr -mah  & 12 & 17 & 41.38\\ \hline
1st & 77 & -endp -aggr -cheb  & 12 & 17 & 41.38\\ \hline
1st & 78 & -high -lpc -diff  & 11 & 18 & 37.93\\ \hline
1st & 79 & -band -aggr -eucl  & 11 & 18 & 37.93\\ \hline
1st & 80 & -endp -fft -mink  & 11 & 18 & 37.93\\ \hline
1st & 81 & -band -aggr -cheb  & 11 & 18 & 37.93\\ \hline
1st & 82 & -band -lpc -diff  & 11 & 18 & 37.93\\ \hline
1st & 83 & -band -aggr -mah  & 11 & 18 & 37.93\\ \hline
1st & 84 & -band -fft -mah  & 11 & 18 & 37.93\\ \hline
1st & 85 & -band -fft -eucl  & 11 & 18 & 37.93\\ \hline
1st & 86 & -endp -aggr -mink  & 11 & 18 & 37.93\\ \hline
1st & 87 & -high -fft -diff  & 11 & 18 & 37.93\\ \hline
1st & 88 & -high -lpc -mink  & 10 & 19 & 34.48\\ \hline
1st & 89 & -raw -minmax -mink  & 10 & 19 & 34.48\\ \hline
\end{tabular}
\end{minipage}
\caption{Consolidated results, Part 3.}
\label{tab:results3}
\end{table}

\begin{table}
\begin{minipage}[b]{\textwidth}
\centering
\begin{tabular}{|c|c|l|c|c|r|} \hline
Guess & Run \# & Configuration & GOOD & BAD & Recognition Rate,\%\\ \hline\hline
1st & 90 & -raw -minmax -eucl  & 10 & 19 & 34.48\\ \hline
1st & 91 & -band -aggr -mink  & 10 & 19 & 34.48\\ \hline
1st & 92 & -raw -minmax -mah  & 10 & 19 & 34.48\\ \hline
1st & 93 & -endp -minmax -eucl  & 10 & 19 & 34.48\\ \hline
1st & 94 & -endp -minmax -cheb  & 10 & 19 & 34.48\\ \hline
1st & 95 & -endp -minmax -mah  & 10 & 19 & 34.48\\ \hline
1st & 96 & -band -fft -cheb  & 10 & 19 & 34.48\\ \hline
1st & 97 & -raw -minmax -cheb  & 9 & 20 & 31.03\\ \hline
1st & 98 & -endp -lpc -nn  & 9 & 20 & 31.03\\ \hline
1st & 99 & -endp -lpc -diff  & 9 & 20 & 31.03\\ \hline
1st & 100 & -endp -minmax -mink  & 8 & 21 & 27.59\\ \hline
1st & 101 & -endp -randfe -diff  & 7 & 22 & 24.14\\ \hline
1st & 102 & -endp -randfe -cheb  & 7 & 22 & 24.14\\ \hline
1st & 103 & -endp -minmax -diff  & 7 & 22 & 24.14\\ \hline
1st & 104 & -endp -minmax -nn  & 7 & 22 & 24.14\\ \hline
1st & 105 & -band -fft -mink  & 7 & 22 & 24.14\\ \hline
1st & 106 & -norm -randfe -eucl  & 6 & 23 & 20.69\\ \hline
1st & 107 & -raw -minmax -diff  & 6 & 23 & 20.69\\ \hline
1st & 108 & -endp -randfe -eucl  & 6 & 23 & 20.69\\ \hline
1st & 109 & -endp -randfe -mah  & 6 & 23 & 20.69\\ \hline
1st & 110 & -low -randfe -mink  & 6 & 23 & 20.69\\ \hline
1st & 111 & -norm -randfe -mah  & 6 & 23 & 20.69\\ \hline
1st & 112 & -norm -minmax -eucl  & 6 & 23 & 20.69\\ \hline
1st & 113 & -norm -minmax -cheb  & 6 & 23 & 20.69\\ \hline
1st & 114 & -low -minmax -mink  & 6 & 23 & 20.69\\ \hline
1st & 115 & -norm -minmax -mah  & 6 & 23 & 20.69\\ \hline
1st & 116 & -norm -randfe -mink  & 6 & 23 & 20.69\\ \hline
1st & 117 & -endp -randfe -mink  & 5 & 24 & 17.24\\ \hline
1st & 118 & -low -minmax -mah  & 5 & 24 & 17.24\\ \hline
1st & 119 & -low -randfe -eucl  & 5 & 24 & 17.24\\ \hline
\end{tabular}
\end{minipage}
\caption{Consolidated results, Part 4.}
\label{tab:results4}
\end{table}

\begin{table}
\begin{minipage}[b]{\textwidth}
\centering
\begin{tabular}{|c|c|l|c|c|r|} \hline
Guess & Run \# & Configuration & GOOD & BAD & Recognition Rate,\%\\ \hline\hline
1st & 120 & -high -minmax -mah  & 5 & 24 & 17.24\\ \hline
1st & 121 & -raw -randfe -mink  & 5 & 24 & 17.24\\ \hline
1st & 122 & -low -randfe -mah  & 5 & 24 & 17.24\\ \hline
1st & 123 & -low -minmax -diff  & 5 & 24 & 17.24\\ \hline
1st & 124 & -high -minmax -diff  & 5 & 24 & 17.24\\ \hline
1st & 125 & -high -minmax -eucl  & 5 & 24 & 17.24\\ \hline
1st & 126 & -low -minmax -eucl  & 5 & 24 & 17.24\\ \hline
1st & 127 & -low -minmax -cheb  & 5 & 24 & 17.24\\ \hline
1st & 128 & -norm -randfe -diff  & 4 & 25 & 13.79\\ \hline
1st & 129 & -norm -randfe -cheb  & 4 & 25 & 13.79\\ \hline
1st & 130 & -high -randfe -mink  & 4 & 25 & 13.79\\ \hline
1st & 131 & -low -randfe -diff  & 4 & 25 & 13.79\\ \hline
1st & 132 & -high -randfe -diff  & 4 & 25 & 13.79\\ \hline
1st & 133 & -high -randfe -eucl  & 4 & 25 & 13.79\\ \hline
1st & 134 & -high -randfe -cheb  & 4 & 25 & 13.79\\ \hline
1st & 135 & -low -randfe -cheb  & 4 & 25 & 13.79\\ \hline
1st & 136 & -norm -minmax -mink  & 4 & 25 & 13.79\\ \hline
1st & 137 & -raw -randfe -eucl  & 4 & 25 & 13.79\\ \hline
1st & 138 & -band -lpc -nn  & 4 & 25 & 13.79\\ \hline
1st & 139 & -raw -randfe -mah  & 4 & 25 & 13.79\\ \hline
1st & 140 & -high -minmax -mink  & 4 & 25 & 13.79\\ \hline
1st & 141 & -raw -minmax -nn  & 4 & 25 & 13.79\\ \hline
1st & 142 & -high -randfe -mah  & 4 & 25 & 13.79\\ \hline
1st & 143 & -high -minmax -cheb  & 4 & 25 & 13.79\\ \hline
1st & 144 & -high -minmax -nn  & 4 & 25 & 13.79\\ \hline
1st & 145 & -endp -randfe -randcl  & 4 & 25 & 13.79\\ \hline
1st & 146 & -band -minmax -mink  & 3 & 26 & 10.34\\ \hline
1st & 147 & -band -minmax -diff  & 3 & 26 & 10.34\\ \hline
1st & 148 & -band -minmax -eucl  & 3 & 26 & 10.34\\ \hline
1st & 149 & -band -minmax -mah  & 3 & 26 & 10.34\\ \hline
\end{tabular}
\end{minipage}
\caption{Consolidated results, Part 5.}
\label{tab:results5}
\end{table}

\begin{table}
\begin{minipage}[b]{\textwidth}
\centering
\begin{tabular}{|c|c|l|c|c|r|} \hline
Guess & Run \# & Configuration & GOOD & BAD & Recognition Rate,\%\\ \hline\hline
1st & 150 & -raw -minmax -randcl  & 3 & 26 & 10.34\\ \hline
1st & 151 & -band -minmax -cheb  & 3 & 26 & 10.34\\ \hline
1st & 152 & -low -lpc -nn  & 3 & 26 & 10.34\\ \hline
1st & 153 & -raw -randfe -diff  & 3 & 26 & 10.34\\ \hline
1st & 154 & -norm -minmax -diff  & 3 & 26 & 10.34\\ \hline
1st & 155 & -boost -lpc -randcl  & 3 & 26 & 10.34\\ \hline
1st & 156 & -raw -randfe -cheb  & 3 & 26 & 10.34\\ \hline
1st & 157 & -boost -minmax -nn  & 3 & 26 & 10.34\\ \hline
1st & 158 & -highpassboost -lpc -nn  & 3 & 26 & 10.34\\ \hline
1st & 159 & -norm -minmax -nn  & 3 & 26 & 10.34\\ \hline
1st & 160 & -highpassboost -minmax -nn  & 3 & 26 & 10.34\\ \hline
1st & 161 & -boost -minmax -randcl  & 3 & 26 & 10.34\\ \hline
1st & 162 & -boost -lpc -nn  & 3 & 26 & 10.34\\ \hline
1st & 163 & -raw -aggr -randcl  & 2 & 27 & 6.90\\ \hline
1st & 164 & -band -randfe -mah  & 2 & 27 & 6.90\\ \hline
1st & 165 & -highpassboost -lpc -randcl  & 2 & 27 & 6.90\\ \hline
1st & 166 & -band -randfe -diff  & 2 & 27 & 6.90\\ \hline
1st & 167 & -band -randfe -eucl  & 2 & 27 & 6.90\\ \hline
1st & 168 & -low -minmax -nn  & 2 & 27 & 6.90\\ \hline
1st & 169 & -boost -randfe -randcl  & 2 & 27 & 6.90\\ \hline
1st & 170 & -band -randfe -cheb  & 2 & 27 & 6.90\\ \hline
1st & 171 & -raw -lpc -randcl  & 2 & 27 & 6.90\\ \hline
1st & 172 & -highpassboost -aggr -randcl  & 2 & 27 & 6.90\\ \hline
1st & 173 & -boost -fft -randcl  & 2 & 27 & 6.90\\ \hline
1st & 174 & -highpassboost -minmax -diff  & 1 & 28 & 3.45\\ \hline
1st & 175 & -boost -randfe -eucl  & 1 & 28 & 3.45\\ \hline
1st & 176 & -highpassboost -minmax -eucl  & 1 & 28 & 3.45\\ \hline
1st & 177 & -boost -lpc -mink  & 1 & 28 & 3.45\\ \hline
1st & 178 & -boost -lpc -diff  & 1 & 28 & 3.45\\ \hline
1st & 179 & -boost -fft -mah  & 1 & 28 & 3.45\\ \hline
\end{tabular}
\end{minipage}
\caption{Consolidated results, Part 6.}
\label{tab:results6}
\end{table}

\begin{table}
\begin{minipage}[b]{\textwidth}
\centering
\begin{tabular}{|c|c|l|c|c|r|} \hline
Guess & Run \# & Configuration & GOOD & BAD & Recognition Rate,\%\\ \hline\hline
1st & 180 & -boost -lpc -eucl  & 1 & 28 & 3.45\\ \hline
1st & 181 & -low -fft -randcl  & 1 & 28 & 3.45\\ \hline
1st & 182 & -low -minmax -randcl  & 1 & 28 & 3.45\\ \hline
1st & 183 & -boost -minmax -mah  & 1 & 28 & 3.45\\ \hline
1st & 184 & -highpassboost -minmax -mah  & 1 & 28 & 3.45\\ \hline
1st & 185 & -boost -randfe -cheb  & 1 & 28 & 3.45\\ \hline
1st & 186 & -high -randfe -randcl  & 1 & 28 & 3.45\\ \hline
1st & 187 & -highpassboost -minmax -cheb  & 1 & 28 & 3.45\\ \hline
1st & 188 & -highpassboost -fft -mah  & 1 & 28 & 3.45\\ \hline
1st & 189 & -boost -lpc -cheb  & 1 & 28 & 3.45\\ \hline
1st & 190 & -boost -aggr -mah  & 1 & 28 & 3.45\\ \hline
1st & 191 & -highpassboost -lpc -mink  & 1 & 28 & 3.45\\ \hline
1st & 192 & -highpassboost -lpc -diff  & 1 & 28 & 3.45\\ \hline
1st & 193 & -endp -lpc -randcl  & 1 & 28 & 3.45\\ \hline
1st & 194 & -highpassboost -lpc -eucl  & 1 & 28 & 3.45\\ \hline
1st & 195 & -high -minmax -randcl  & 1 & 28 & 3.45\\ \hline
1st & 196 & -highpassboost -lpc -cheb  & 1 & 28 & 3.45\\ \hline
1st & 197 & -norm -fft -randcl  & 1 & 28 & 3.45\\ \hline
1st & 198 & -band -aggr -randcl  & 1 & 28 & 3.45\\ \hline
1st & 199 & -low -randfe -randcl  & 1 & 28 & 3.45\\ \hline
1st & 200 & -boost -aggr -mink  & 1 & 28 & 3.45\\ \hline
1st & 201 & -boost -aggr -diff  & 1 & 28 & 3.45\\ \hline
1st & 202 & -endp -aggr -randcl  & 1 & 28 & 3.45\\ \hline
1st & 203 & -boost -aggr -eucl  & 1 & 28 & 3.45\\ \hline
1st & 204 & -boost -fft -mink  & 1 & 28 & 3.45\\ \hline
1st & 205 & -boost -randfe -mah  & 1 & 28 & 3.45\\ \hline
1st & 206 & -boost -fft -diff  & 1 & 28 & 3.45\\ \hline
1st & 207 & -boost -fft -eucl  & 1 & 28 & 3.45\\ \hline
1st & 208 & -highpassboost -randfe -mink  & 1 & 28 & 3.45\\ \hline
1st & 209 & -highpassboost -randfe -diff  & 1 & 28 & 3.45\\ \hline
\end{tabular}
\end{minipage}
\caption{Consolidated results, Part 7.}
\label{tab:results7}
\end{table}

\begin{table}
\begin{minipage}[b]{\textwidth}
\centering
\begin{tabular}{|c|c|l|c|c|r|} \hline
Guess & Run \# & Configuration & GOOD & BAD & Recognition Rate,\%\\ \hline\hline
1st & 210 & -boost -minmax -mink  & 1 & 28 & 3.45\\ \hline
1st & 211 & -boost -minmax -diff  & 1 & 28 & 3.45\\ \hline
1st & 212 & -highpassboost -randfe -eucl  & 1 & 28 & 3.45\\ \hline
1st & 213 & -boost -minmax -eucl  & 1 & 28 & 3.45\\ \hline
1st & 214 & -low -aggr -randcl  & 1 & 28 & 3.45\\ \hline
1st & 215 & -band -fft -randcl  & 1 & 28 & 3.45\\ \hline
1st & 216 & -boost -aggr -cheb  & 1 & 28 & 3.45\\ \hline
1st & 217 & -band -randfe -randcl  & 1 & 28 & 3.45\\ \hline
1st & 218 & -boost -fft -cheb  & 1 & 28 & 3.45\\ \hline
1st & 219 & -highpassboost -aggr -mink  & 1 & 28 & 3.45\\ \hline
1st & 220 & -highpassboost -aggr -diff  & 1 & 28 & 3.45\\ \hline
1st & 221 & -highpassboost -fft -mink  & 1 & 28 & 3.45\\ \hline
1st & 222 & -endp -minmax -randcl  & 1 & 28 & 3.45\\ \hline
1st & 223 & -highpassboost -fft -diff  & 1 & 28 & 3.45\\ \hline
1st & 224 & -highpassboost -aggr -eucl  & 1 & 28 & 3.45\\ \hline
1st & 225 & -highpassboost -randfe -cheb  & 1 & 28 & 3.45\\ \hline
1st & 226 & -high -lpc -nn  & 1 & 28 & 3.45\\ \hline
1st & 227 & -boost -minmax -cheb  & 1 & 28 & 3.45\\ \hline
1st & 228 & -highpassboost -fft -eucl  & 1 & 28 & 3.45\\ \hline
1st & 229 & -boost -lpc -mah  & 1 & 28 & 3.45\\ \hline
1st & 230 & -norm -randfe -randcl  & 1 & 28 & 3.45\\ \hline
1st & 231 & -highpassboost -aggr -cheb  & 1 & 28 & 3.45\\ \hline
1st & 232 & -highpassboost -fft -cheb  & 1 & 28 & 3.45\\ \hline
1st & 233 & -band -minmax -randcl  & 1 & 28 & 3.45\\ \hline
1st & 234 & -boost -aggr -randcl  & 1 & 28 & 3.45\\ \hline
1st & 235 & -highpassboost -lpc -mah  & 1 & 28 & 3.45\\ \hline
1st & 236 & -highpassboost -aggr -mah  & 1 & 28 & 3.45\\ \hline
1st & 237 & -high -lpc -randcl  & 1 & 28 & 3.45\\ \hline
1st & 238 & -highpassboost -randfe -mah  & 1 & 28 & 3.45\\ \hline
1st & 239 & -boost -randfe -mink  & 1 & 28 & 3.45\\ \hline
\end{tabular}
\end{minipage}
\caption{Consolidated results, Part 8.}
\label{tab:results8}
\end{table}

\begin{table}
\begin{minipage}[b]{\textwidth}
\centering
\begin{tabular}{|c|c|l|c|c|r|} \hline
Guess & Run \# & Configuration & GOOD & BAD & Recognition Rate,\%\\ \hline\hline
1st & 240 & -boost -randfe -diff  & 1 & 28 & 3.45\\ \hline
1st & 241 & -highpassboost -minmax -mink  & 1 & 28 & 3.45\\ \hline
1st & 242 & -raw -randfe -randcl  & 0 & 29 & 0.00\\ \hline
1st & 243 & -highpassboost -fft -randcl  & 0 & 29 & 0.00\\ \hline
1st & 244 & -band -lpc -randcl  & 0 & 29 & 0.00\\ \hline
1st & 245 & -endp -fft -randcl  & 0 & 29 & 0.00\\ \hline
1st & 246 & -raw -fft -randcl  & 0 & 29 & 0.00\\ \hline
1st & 247 & -norm -lpc -randcl  & 0 & 29 & 0.00\\ \hline
1st & 248 & -highpassboost -randfe -randcl  & 0 & 29 & 0.00\\ \hline
1st & 249 & -high -aggr -randcl  & 0 & 29 & 0.00\\ \hline
1st & 250 & -band -randfe -mink  & 0 & 29 & 0.00\\ \hline
1st & 251 & -low -lpc -randcl  & 0 & 29 & 0.00\\ \hline
1st & 252 & -highpassboost -minmax -randcl  & 0 & 29 & 0.00\\ \hline
1st & 253 & -norm -aggr -randcl  & 0 & 29 & 0.00\\ \hline
1st & 254 & -high -fft -randcl  & 0 & 29 & 0.00\\ \hline
1st & 255 & -band -minmax -nn  & 0 & 29 & 0.00\\ \hline
1st & 256 & -norm -minmax -randcl  & 0 & 29 & 0.00\\ \hline
2nd & 1 & -endp -lpc -cheb  & 24 & 5 & 82.76\\ \hline
2nd & 2 & -raw -aggr -eucl  & 24 & 5 & 82.76\\ \hline
2nd & 3 & -norm -aggr -diff  & 24 & 5 & 82.76\\ \hline
2nd & 4 & -norm -aggr -cheb  & 24 & 5 & 82.76\\ \hline
2nd & 5 & -raw -aggr -mah  & 24 & 5 & 82.76\\ \hline
2nd & 6 & -raw -fft -mah  & 24 & 5 & 82.76\\ \hline
2nd & 7 & -raw -fft -eucl  & 24 & 5 & 82.76\\ \hline
2nd & 8 & -norm -fft -diff  & 24 & 5 & 82.76\\ \hline
2nd & 9 & -norm -fft -cheb  & 24 & 5 & 82.76\\ \hline
2nd & 10 & -raw -aggr -cheb  & 22 & 7 & 75.86\\ \hline
2nd & 11 & -endp -lpc -mah  & 22 & 7 & 75.86\\ \hline
2nd & 12 & -endp -lpc -eucl  & 22 & 7 & 75.86\\ \hline
2nd & 13 & -raw -fft -mink  & 25 & 4 & 86.21\\ \hline
\end{tabular}
\end{minipage}
\caption{Consolidated results, Part 9.}
\label{tab:results9}
\end{table}

\begin{table}
\begin{minipage}[b]{\textwidth}
\centering
\begin{tabular}{|c|c|l|c|c|r|} \hline
Guess & Run \# & Configuration & GOOD & BAD & Recognition Rate,\%\\ \hline\hline
2nd & 14 & -norm -fft -mah  & 24 & 5 & 82.76\\ \hline
2nd & 15 & -norm -fft -eucl  & 24 & 5 & 82.76\\ \hline
2nd & 16 & -norm -aggr -eucl  & 24 & 5 & 82.76\\ \hline
2nd & 17 & -low -aggr -diff  & 21 & 8 & 72.41\\ \hline
2nd & 18 & -norm -aggr -mah  & 24 & 5 & 82.76\\ \hline
2nd & 19 & -raw -fft -cheb  & 22 & 7 & 75.86\\ \hline
2nd & 20 & -raw -aggr -mink  & 25 & 4 & 86.21\\ \hline
2nd & 21 & -norm -aggr -mink  & 24 & 5 & 82.76\\ \hline
2nd & 22 & -low -fft -cheb  & 22 & 7 & 75.86\\ \hline
2nd & 23 & -raw -lpc -mink  & 23 & 6 & 79.31\\ \hline
2nd & 24 & -raw -lpc -diff  & 23 & 6 & 79.31\\ \hline
2nd & 25 & -raw -lpc -eucl  & 23 & 6 & 79.31\\ \hline
2nd & 26 & -raw -lpc -mah  & 23 & 6 & 79.31\\ \hline
2nd & 27 & -raw -lpc -cheb  & 23 & 6 & 79.31\\ \hline
2nd & 28 & -low -aggr -eucl  & 23 & 6 & 79.31\\ \hline
2nd & 29 & -norm -lpc -mah  & 23 & 6 & 79.31\\ \hline
2nd & 30 & -norm -lpc -mink  & 23 & 6 & 79.31\\ \hline
2nd & 31 & -norm -lpc -diff  & 23 & 6 & 79.31\\ \hline
2nd & 32 & -norm -lpc -eucl  & 23 & 6 & 79.31\\ \hline
2nd & 33 & -low -aggr -cheb  & 22 & 7 & 75.86\\ \hline
2nd & 34 & -norm -lpc -cheb  & 23 & 6 & 79.31\\ \hline
2nd & 35 & -low -aggr -mah  & 23 & 6 & 79.31\\ \hline
2nd & 36 & -low -fft -mah  & 23 & 6 & 79.31\\ \hline
2nd & 37 & -norm -fft -mink  & 24 & 5 & 82.76\\ \hline
2nd & 38 & -low -fft -diff  & 21 & 8 & 72.41\\ \hline
2nd & 39 & -low -fft -eucl  & 23 & 6 & 79.31\\ \hline
2nd & 40 & -raw -aggr -diff  & 22 & 7 & 75.86\\ \hline
2nd & 41 & -high -aggr -mink  & 21 & 8 & 72.41\\ \hline
2nd & 42 & -high -aggr -eucl  & 21 & 8 & 72.41\\ \hline
2nd & 43 & -endp -lpc -mink  & 20 & 9 & 68.97\\ \hline
\end{tabular}
\end{minipage}
\caption{Consolidated results, Part 10.}
\label{tab:results10}
\end{table}

\begin{table}
\begin{minipage}[b]{\textwidth}
\centering
\begin{tabular}{|c|c|l|c|c|r|} \hline
Guess & Run \# & Configuration & GOOD & BAD & Recognition Rate,\%\\ \hline\hline
2nd & 44 & -high -aggr -mah  & 21 & 8 & 72.41\\ \hline
2nd & 45 & -raw -fft -diff  & 22 & 7 & 75.86\\ \hline
2nd & 46 & -high -aggr -cheb  & 20 & 9 & 68.97\\ \hline
2nd & 47 & -low -aggr -mink  & 24 & 5 & 82.76\\ \hline
2nd & 48 & -low -fft -mink  & 24 & 5 & 82.76\\ \hline
2nd & 49 & -high -fft -cheb  & 20 & 9 & 68.97\\ \hline
2nd & 50 & -high -fft -mah  & 21 & 8 & 72.41\\ \hline
2nd & 51 & -low -lpc -cheb  & 22 & 7 & 75.86\\ \hline
2nd & 52 & -high -fft -mink  & 19 & 10 & 65.52\\ \hline
2nd & 53 & -high -fft -eucl  & 21 & 8 & 72.41\\ \hline
2nd & 54 & -low -lpc -eucl  & 20 & 9 & 68.97\\ \hline
2nd & 55 & -low -lpc -mah  & 20 & 9 & 68.97\\ \hline
2nd & 56 & -low -lpc -mink  & 18 & 11 & 62.07\\ \hline
2nd & 57 & -low -lpc -diff  & 21 & 8 & 72.41\\ \hline
2nd & 58 & -high -lpc -cheb  & 19 & 10 & 65.52\\ \hline
2nd & 59 & -raw -lpc -nn  & 14 & 15 & 48.28\\ \hline
2nd & 60 & -band -aggr -diff  & 14 & 15 & 48.28\\ \hline
2nd & 61 & -norm -lpc -nn  & 15 & 14 & 51.72\\ \hline
2nd & 62 & -band -fft -diff  & 14 & 15 & 48.28\\ \hline
2nd & 63 & -high -lpc -eucl  & 17 & 12 & 58.62\\ \hline
2nd & 64 & -high -aggr -diff  & 18 & 11 & 62.07\\ \hline
2nd & 65 & -endp -fft -diff  & 17 & 12 & 58.62\\ \hline
2nd & 66 & -endp -fft -eucl  & 16 & 13 & 55.17\\ \hline
2nd & 67 & -band -lpc -mink  & 16 & 13 & 55.17\\ \hline
2nd & 68 & -band -lpc -mah  & 16 & 13 & 55.17\\ \hline
2nd & 69 & -band -lpc -eucl  & 16 & 13 & 55.17\\ \hline
2nd & 70 & -endp -fft -cheb  & 17 & 12 & 58.62\\ \hline
2nd & 71 & -band -lpc -cheb  & 17 & 12 & 58.62\\ \hline
2nd & 72 & -endp -fft -mah  & 16 & 13 & 55.17\\ \hline
2nd & 73 & -high -lpc -mah  & 17 & 12 & 58.62\\ \hline
\end{tabular}
\end{minipage}
\caption{Consolidated results, Part 11.}
\label{tab:results11}
\end{table}

\begin{table}
\begin{minipage}[b]{\textwidth}
\centering
\begin{tabular}{|c|c|l|c|c|r|} \hline
Guess & Run \# & Configuration & GOOD & BAD & Recognition Rate,\%\\ \hline\hline
2nd & 74 & -endp -aggr -diff  & 18 & 11 & 62.07\\ \hline
2nd & 75 & -endp -aggr -eucl  & 17 & 12 & 58.62\\ \hline
2nd & 76 & -endp -aggr -mah  & 17 & 12 & 58.62\\ \hline
2nd & 77 & -endp -aggr -cheb  & 18 & 11 & 62.07\\ \hline
2nd & 78 & -high -lpc -diff  & 18 & 11 & 62.07\\ \hline
2nd & 79 & -band -aggr -eucl  & 15 & 14 & 51.72\\ \hline
2nd & 80 & -endp -fft -mink  & 14 & 15 & 48.28\\ \hline
2nd & 81 & -band -aggr -cheb  & 14 & 15 & 48.28\\ \hline
2nd & 82 & -band -lpc -diff  & 17 & 12 & 58.62\\ \hline
2nd & 83 & -band -aggr -mah  & 15 & 14 & 51.72\\ \hline
2nd & 84 & -band -fft -mah  & 15 & 14 & 51.72\\ \hline
2nd & 85 & -band -fft -eucl  & 15 & 14 & 51.72\\ \hline
2nd & 86 & -endp -aggr -mink  & 14 & 15 & 48.28\\ \hline
2nd & 87 & -high -fft -diff  & 18 & 11 & 62.07\\ \hline
2nd & 88 & -high -lpc -mink  & 14 & 15 & 48.28\\ \hline
2nd & 89 & -raw -minmax -mink  & 12 & 17 & 41.38\\ \hline
2nd & 90 & -raw -minmax -eucl  & 12 & 17 & 41.38\\ \hline
2nd & 91 & -band -aggr -mink  & 13 & 16 & 44.83\\ \hline
2nd & 92 & -raw -minmax -mah  & 12 & 17 & 41.38\\ \hline
2nd & 93 & -endp -minmax -eucl  & 12 & 17 & 41.38\\ \hline
2nd & 94 & -endp -minmax -cheb  & 12 & 17 & 41.38\\ \hline
2nd & 95 & -endp -minmax -mah  & 12 & 17 & 41.38\\ \hline
2nd & 96 & -band -fft -cheb  & 14 & 15 & 48.28\\ \hline
2nd & 97 & -raw -minmax -cheb  & 11 & 18 & 37.93\\ \hline
2nd & 98 & -endp -lpc -nn  & 12 & 17 & 41.38\\ \hline
2nd & 99 & -endp -lpc -diff  & 19 & 10 & 65.52\\ \hline
2nd & 100 & -endp -minmax -mink  & 12 & 17 & 41.38\\ \hline
2nd & 101 & -endp -randfe -diff  & 8 & 21 & 27.59\\ \hline
2nd & 102 & -endp -randfe -cheb  & 8 & 21 & 27.59\\ \hline
2nd & 103 & -endp -minmax -diff  & 12 & 17 & 41.38\\ \hline
\end{tabular}
\end{minipage}
\caption{Consolidated results, Part 12.}
\label{tab:results12}
\end{table}

\begin{table}
\begin{minipage}[b]{\textwidth}
\centering
\begin{tabular}{|c|c|l|c|c|r|} \hline
Guess & Run \# & Configuration & GOOD & BAD & Recognition Rate,\%\\ \hline\hline
2nd & 104 & -endp -minmax -nn  & 7 & 22 & 24.14\\ \hline
2nd & 105 & -band -fft -mink  & 13 & 16 & 44.83\\ \hline
2nd & 106 & -norm -randfe -eucl  & 10 & 19 & 34.48\\ \hline
2nd & 107 & -raw -minmax -diff  & 10 & 19 & 34.48\\ \hline
2nd & 108 & -endp -randfe -eucl  & 9 & 20 & 31.03\\ \hline
2nd & 109 & -endp -randfe -mah  & 9 & 20 & 31.03\\ \hline
2nd & 110 & -low -randfe -mink  & 9 & 20 & 31.03\\ \hline
2nd & 111 & -norm -randfe -mah  & 10 & 19 & 34.48\\ \hline
2nd & 112 & -norm -minmax -eucl  & 9 & 20 & 31.03\\ \hline
2nd & 113 & -norm -minmax -cheb  & 10 & 19 & 34.48\\ \hline
2nd & 114 & -low -minmax -mink  & 8 & 21 & 27.59\\ \hline
2nd & 115 & -norm -minmax -mah  & 9 & 20 & 31.03\\ \hline
2nd & 116 & -norm -randfe -mink  & 10 & 19 & 34.48\\ \hline
2nd & 117 & -endp -randfe -mink  & 8 & 21 & 27.59\\ \hline
2nd & 118 & -low -minmax -mah  & 6 & 23 & 20.69\\ \hline
2nd & 119 & -low -randfe -eucl  & 9 & 20 & 31.03\\ \hline
2nd & 120 & -high -minmax -mah  & 6 & 23 & 20.69\\ \hline
2nd & 121 & -raw -randfe -mink  & 9 & 20 & 31.03\\ \hline
2nd & 122 & -low -randfe -mah  & 9 & 20 & 31.03\\ \hline
2nd & 123 & -low -minmax -diff  & 6 & 23 & 20.69\\ \hline
2nd & 124 & -high -minmax -diff  & 6 & 23 & 20.69\\ \hline
2nd & 125 & -high -minmax -eucl  & 6 & 23 & 20.69\\ \hline
2nd & 126 & -low -minmax -eucl  & 6 & 23 & 20.69\\ \hline
2nd & 127 & -low -minmax -cheb  & 6 & 23 & 20.69\\ \hline
2nd & 128 & -norm -randfe -diff  & 10 & 19 & 34.48\\ \hline
2nd & 129 & -norm -randfe -cheb  & 10 & 19 & 34.48\\ \hline
2nd & 130 & -high -randfe -mink  & 5 & 24 & 17.24\\ \hline
2nd & 131 & -low -randfe -diff  & 9 & 20 & 31.03\\ \hline
2nd & 132 & -high -randfe -diff  & 7 & 22 & 24.14\\ \hline
2nd & 133 & -high -randfe -eucl  & 6 & 23 & 20.69\\ \hline
\end{tabular}
\end{minipage}
\caption{Consolidated results, Part 13.}
\label{tab:results13}
\end{table}

\begin{table}
\begin{minipage}[b]{\textwidth}
\centering
\begin{tabular}{|c|c|l|c|c|r|} \hline
Guess & Run \# & Configuration & GOOD & BAD & Recognition Rate,\%\\ \hline\hline
2nd & 134 & -high -randfe -cheb  & 7 & 22 & 24.14\\ \hline
2nd & 135 & -low -randfe -cheb  & 9 & 20 & 31.03\\ \hline
2nd & 136 & -norm -minmax -mink  & 10 & 19 & 34.48\\ \hline
2nd & 137 & -raw -randfe -eucl  & 8 & 21 & 27.59\\ \hline
2nd & 138 & -band -lpc -nn  & 4 & 25 & 13.79\\ \hline
2nd & 139 & -raw -randfe -mah  & 8 & 21 & 27.59\\ \hline
2nd & 140 & -high -minmax -mink  & 6 & 23 & 20.69\\ \hline
2nd & 141 & -raw -minmax -nn  & 4 & 25 & 13.79\\ \hline
2nd & 142 & -high -randfe -mah  & 6 & 23 & 20.69\\ \hline
2nd & 143 & -high -minmax -cheb  & 5 & 24 & 17.24\\ \hline
2nd & 144 & -high -minmax -nn  & 4 & 25 & 13.79\\ \hline
2nd & 145 & -endp -randfe -randcl  & 6 & 23 & 20.69\\ \hline
2nd & 146 & -band -minmax -mink  & 5 & 24 & 17.24\\ \hline
2nd & 147 & -band -minmax -diff  & 5 & 24 & 17.24\\ \hline
2nd & 148 & -band -minmax -eucl  & 5 & 24 & 17.24\\ \hline
2nd & 149 & -band -minmax -mah  & 5 & 24 & 17.24\\ \hline
2nd & 150 & -raw -minmax -randcl  & 3 & 26 & 10.34\\ \hline
2nd & 151 & -band -minmax -cheb  & 4 & 25 & 13.79\\ \hline
2nd & 152 & -low -lpc -nn  & 4 & 25 & 13.79\\ \hline
2nd & 153 & -raw -randfe -diff  & 6 & 23 & 20.69\\ \hline
2nd & 154 & -norm -minmax -diff  & 9 & 20 & 31.03\\ \hline
2nd & 155 & -boost -lpc -randcl  & 3 & 26 & 10.34\\ \hline
2nd & 156 & -raw -randfe -cheb  & 6 & 23 & 20.69\\ \hline
2nd & 157 & -boost -minmax -nn  & 4 & 25 & 13.79\\ \hline
2nd & 158 & -highpassboost -lpc -nn  & 4 & 25 & 13.79\\ \hline
2nd & 159 & -norm -minmax -nn  & 3 & 26 & 10.34\\ \hline
2nd & 160 & -highpassboost -minmax -nn  & 4 & 25 & 13.79\\ \hline
2nd & 161 & -boost -minmax -randcl  & 5 & 24 & 17.24\\ \hline
2nd & 162 & -boost -lpc -nn  & 4 & 25 & 13.79\\ \hline
2nd & 163 & -raw -aggr -randcl  & 4 & 25 & 13.79\\ \hline
\end{tabular}
\end{minipage}
\caption{Consolidated results, Part 14.}
\label{tab:results14}
\end{table}

\begin{table}
\begin{minipage}[b]{\textwidth}
\centering
\begin{tabular}{|c|c|l|c|c|r|} \hline
Guess & Run \# & Configuration & GOOD & BAD & Recognition Rate,\%\\ \hline\hline
2nd & 164 & -band -randfe -mah  & 5 & 24 & 17.24\\ \hline
2nd & 165 & -highpassboost -lpc -randcl  & 2 & 27 & 6.90\\ \hline
2nd & 166 & -band -randfe -diff  & 4 & 25 & 13.79\\ \hline
2nd & 167 & -band -randfe -eucl  & 5 & 24 & 17.24\\ \hline
2nd & 168 & -low -minmax -nn  & 3 & 26 & 10.34\\ \hline
2nd & 169 & -boost -randfe -randcl  & 3 & 26 & 10.34\\ \hline
2nd & 170 & -band -randfe -cheb  & 4 & 25 & 13.79\\ \hline
2nd & 171 & -raw -lpc -randcl  & 3 & 26 & 10.34\\ \hline
2nd & 172 & -highpassboost -aggr -randcl  & 2 & 27 & 6.90\\ \hline
2nd & 173 & -boost -fft -randcl  & 3 & 26 & 10.34\\ \hline
2nd & 174 & -highpassboost -minmax -diff  & 2 & 27 & 6.90\\ \hline
2nd & 175 & -boost -randfe -eucl  & 2 & 27 & 6.90\\ \hline
2nd & 176 & -highpassboost -minmax -eucl  & 2 & 27 & 6.90\\ \hline
2nd & 177 & -boost -lpc -mink  & 2 & 27 & 6.90\\ \hline
2nd & 178 & -boost -lpc -diff  & 2 & 27 & 6.90\\ \hline
2nd & 179 & -boost -fft -mah  & 2 & 27 & 6.90\\ \hline
2nd & 180 & -boost -lpc -eucl  & 2 & 27 & 6.90\\ \hline
2nd & 181 & -low -fft -randcl  & 3 & 26 & 10.34\\ \hline
2nd & 182 & -low -minmax -randcl  & 2 & 27 & 6.90\\ \hline
2nd & 183 & -boost -minmax -mah  & 2 & 27 & 6.90\\ \hline
2nd & 184 & -highpassboost -minmax -mah  & 2 & 27 & 6.90\\ \hline
2nd & 185 & -boost -randfe -cheb  & 2 & 27 & 6.90\\ \hline
2nd & 186 & -high -randfe -randcl  & 2 & 27 & 6.90\\ \hline
2nd & 187 & -highpassboost -minmax -cheb  & 2 & 27 & 6.90\\ \hline
2nd & 188 & -highpassboost -fft -mah  & 2 & 27 & 6.90\\ \hline
2nd & 189 & -boost -lpc -cheb  & 2 & 27 & 6.90\\ \hline
2nd & 190 & -boost -aggr -mah  & 2 & 27 & 6.90\\ \hline
2nd & 191 & -highpassboost -lpc -mink  & 2 & 27 & 6.90\\ \hline
2nd & 192 & -highpassboost -lpc -diff  & 2 & 27 & 6.90\\ \hline
2nd & 193 & -endp -lpc -randcl  & 1 & 28 & 3.45\\ \hline
\end{tabular}
\end{minipage}
\caption{Consolidated results, Part 15.}
\label{tab:results15}
\end{table}

\begin{table}
\begin{minipage}[b]{\textwidth}
\centering
\begin{tabular}{|c|c|l|c|c|r|} \hline
Guess & Run \# & Configuration & GOOD & BAD & Recognition Rate,\%\\ \hline\hline
2nd & 194 & -highpassboost -lpc -eucl  & 2 & 27 & 6.90\\ \hline
2nd & 195 & -high -minmax -randcl  & 1 & 28 & 3.45\\ \hline
2nd & 196 & -highpassboost -lpc -cheb  & 2 & 27 & 6.90\\ \hline
2nd & 197 & -norm -fft -randcl  & 3 & 26 & 10.34\\ \hline
2nd & 198 & -band -aggr -randcl  & 1 & 28 & 3.45\\ \hline
2nd & 199 & -low -randfe -randcl  & 1 & 28 & 3.45\\ \hline
2nd & 200 & -boost -aggr -mink  & 2 & 27 & 6.90\\ \hline
2nd & 201 & -boost -aggr -diff  & 2 & 27 & 6.90\\ \hline
2nd & 202 & -endp -aggr -randcl  & 3 & 26 & 10.34\\ \hline
2nd & 203 & -boost -aggr -eucl  & 2 & 27 & 6.90\\ \hline
2nd & 204 & -boost -fft -mink  & 2 & 27 & 6.90\\ \hline
2nd & 205 & -boost -randfe -mah  & 2 & 27 & 6.90\\ \hline
2nd & 206 & -boost -fft -diff  & 2 & 27 & 6.90\\ \hline
2nd & 207 & -boost -fft -eucl  & 2 & 27 & 6.90\\ \hline
2nd & 208 & -highpassboost -randfe -mink  & 2 & 27 & 6.90\\ \hline
2nd & 209 & -highpassboost -randfe -diff  & 2 & 27 & 6.90\\ \hline
2nd & 210 & -boost -minmax -mink  & 2 & 27 & 6.90\\ \hline
2nd & 211 & -boost -minmax -diff  & 2 & 27 & 6.90\\ \hline
2nd & 212 & -highpassboost -randfe -eucl  & 2 & 27 & 6.90\\ \hline
2nd & 213 & -boost -minmax -eucl  & 2 & 27 & 6.90\\ \hline
2nd & 214 & -low -aggr -randcl  & 2 & 27 & 6.90\\ \hline
2nd & 215 & -band -fft -randcl  & 2 & 27 & 6.90\\ \hline
2nd & 216 & -boost -aggr -cheb  & 2 & 27 & 6.90\\ \hline
2nd & 217 & -band -randfe -randcl  & 2 & 27 & 6.90\\ \hline
2nd & 218 & -boost -fft -cheb  & 2 & 27 & 6.90\\ \hline
2nd & 219 & -highpassboost -aggr -mink  & 2 & 27 & 6.90\\ \hline
2nd & 220 & -highpassboost -aggr -diff  & 2 & 27 & 6.90\\ \hline
2nd & 221 & -highpassboost -fft -mink  & 2 & 27 & 6.90\\ \hline
2nd & 222 & -endp -minmax -randcl  & 2 & 27 & 6.90\\ \hline
2nd & 223 & -highpassboost -fft -diff  & 2 & 27 & 6.90\\ \hline
\end{tabular}
\end{minipage}
\caption{Consolidated results, Part 16.}
\label{tab:results16}
\end{table}

\begin{table}
\begin{minipage}[b]{\textwidth}
\centering
\begin{tabular}{|c|c|l|c|c|r|} \hline
Guess & Run \# & Configuration & GOOD & BAD & Recognition Rate,\%\\ \hline\hline
2nd & 224 & -highpassboost -aggr -eucl  & 2 & 27 & 6.90\\ \hline
2nd & 225 & -highpassboost -randfe -cheb  & 2 & 27 & 6.90\\ \hline
2nd & 226 & -high -lpc -nn  & 2 & 27 & 6.90\\ \hline
2nd & 227 & -boost -minmax -cheb  & 2 & 27 & 6.90\\ \hline
2nd & 228 & -highpassboost -fft -eucl  & 2 & 27 & 6.90\\ \hline
2nd & 229 & -boost -lpc -mah  & 2 & 27 & 6.90\\ \hline
2nd & 230 & -norm -randfe -randcl  & 1 & 28 & 3.45\\ \hline
2nd & 231 & -highpassboost -aggr -cheb  & 2 & 27 & 6.90\\ \hline
2nd & 232 & -highpassboost -fft -cheb  & 2 & 27 & 6.90\\ \hline
2nd & 233 & -band -minmax -randcl  & 2 & 27 & 6.90\\ \hline
2nd & 234 & -boost -aggr -randcl  & 2 & 27 & 6.90\\ \hline
2nd & 235 & -highpassboost -lpc -mah  & 2 & 27 & 6.90\\ \hline
2nd & 236 & -highpassboost -aggr -mah  & 2 & 27 & 6.90\\ \hline
2nd & 237 & -high -lpc -randcl  & 1 & 28 & 3.45\\ \hline
2nd & 238 & -highpassboost -randfe -mah  & 2 & 27 & 6.90\\ \hline
2nd & 239 & -boost -randfe -mink  & 2 & 27 & 6.90\\ \hline
2nd & 240 & -boost -randfe -diff  & 2 & 27 & 6.90\\ \hline
2nd & 241 & -highpassboost -minmax -mink  & 2 & 27 & 6.90\\ \hline
2nd & 242 & -raw -randfe -randcl  & 0 & 29 & 0.00\\ \hline
2nd & 243 & -highpassboost -fft -randcl  & 1 & 28 & 3.45\\ \hline
2nd & 244 & -band -lpc -randcl  & 0 & 29 & 0.00\\ \hline
2nd & 245 & -endp -fft -randcl  & 0 & 29 & 0.00\\ \hline
2nd & 246 & -raw -fft -randcl  & 2 & 27 & 6.90\\ \hline
2nd & 247 & -norm -lpc -randcl  & 1 & 28 & 3.45\\ \hline
2nd & 248 & -highpassboost -randfe -randcl  & 1 & 28 & 3.45\\ \hline
2nd & 249 & -high -aggr -randcl  & 0 & 29 & 0.00\\ \hline
2nd & 250 & -band -randfe -mink  & 1 & 28 & 3.45\\ \hline
2nd & 251 & -low -lpc -randcl  & 2 & 27 & 6.90\\ \hline
2nd & 252 & -highpassboost -minmax -randcl  & 2 & 27 & 6.90\\ \hline
2nd & 253 & -norm -aggr -randcl  & 1 & 28 & 3.45\\ \hline
\end{tabular}
\end{minipage}
\caption{Consolidated results, Part 17.}
\label{tab:results17}
\end{table}

\begin{table}
\begin{minipage}[b]{\textwidth}
\centering
\begin{tabular}{|c|c|l|c|c|r|} \hline
Guess & Run \# & Configuration & GOOD & BAD & Recognition Rate,\%\\ \hline\hline
2nd & 254 & -high -fft -randcl  & 3 & 26 & 10.34\\ \hline
2nd & 255 & -band -minmax -nn  & 1 & 28 & 3.45\\ \hline
2nd & 256 & -norm -minmax -randcl  & 1 & 28 & 3.45\\ \hline
\end{tabular}
\end{minipage}
\caption{Consolidated results, Part 18.}
\label{tab:results18}
\end{table}



% EOF
